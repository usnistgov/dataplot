MENU FILE #10 (STATISTICS)
 
0                500         0
STATISTICS
1.               600         0
Assumptions--What are Underlying Assumptions of a Process?
1.1              600         9
What are the 4 Assumptions?
1.2              600        18
What Gain is Achieved if Assumptions Hold?
1.3              600        25
What Penalty is Incurred if Assumptions Fail?
1.4              600        34
How Test the 4 Underlying Assumptions of a Process?
2                700         0
Process Summarization--How Summarize the Output from a Process?
2.1              700        11
Number of Data Points = ...
2.2              700        19
Estimate of "Typical Value" (Location) = ...
2.3              700        27
Estimate of Precision (Variation) = ...
2.4              700        35
Uncertainty in the "Typical Value" Estimate = ...
2.5              700        44
Verbal Description of How the Uncertainty was Computed
2.6              700        59
"Best-Fit" Distribution = ...
3                800         0
Location Analysis--What is the "Typical Value" of a Process?
3.1              800        11
How Estimate Location?
3.2              800        27
How Attach Uncertainty to Location Estimate?
3.3              800        35
How Compute Confidence Interval for Location Estimate?
3.4              800        48
How Many Units for Prescribed Uncertainty in Loc. Est.?
3.5              800        53
How Test to See if Eng. Mod. has Improved "Typical" Process Yield?
3.5.1            800        61
Graphical Analysis
3.5.2            800        72
Classical Analysis
3.5.3            800        81
Non-parametric Analysis
3.6              800        92
Is "Typical" Process Yield the Same for All Levels of a Factor?
4               1000         0
Variation Analysis--What is the Precision of a Process?
4.1             1000        12
How Estimate Process Precision?
4.2             1000        21
How Attach Uncertainty to Precision Estimate?
4.3             1000        34
How Compute Confidence Interval for Precision Estimate?
4.4             1000        39
How Many Units for Prescribed Uncertainty in Prec. Est.?
4.5             1000        44
How Test to See if Eng. Mod. has Improved Process Precision?
4.5.1           1000        52
Graphical Analysis
4.5.2           1000        63
Classical Analysis
4.5.3           1000        79
Non-parametric Analysis
4.6             1000        86
Is Process Precision the Same for All Levels of a Factor?
4.7             1000       107
How Transform to Fixed Process Precision?
5               1200         0
What Distribution does the Process Follow (Distrib. Analysis)?
5.1             1200        14
What is the General Distributional Shape?
5.2             1200        30
What Distributional Family Best Fits the Data?
5.3             1200        48
What Member of a Distributional Family Best Fits the Data?
5.4             1200        66
What Specific Distribution Best Fits the Data?
5.4.1           1200        76
Is the Data Normal?
5.4.2           1200        86
If Non-normal, how Transform to Normal?
5.4.3           1200        92
Is the Data Weibull?
5.4.4           1200       100
Is the Data Uniform?
5.4.5           1200       107
Is the Data Some Other Distribution?
5.5             1200       154
How Attach Uncertainty to "Best-Fit" Dist. Estimate?
5.6             1200       159
How Compute Confidence Interval for "Best-Fit" Dist. Estimate?
5.7             1200       164
How Many Units for Prescribed Uncertainty in "Best=Fit" Dist.?
5.8             1200       169
How Test to See if Eng. Mod. has Changed "Best-Fit" Distribution?
5.9             1200       181
Is "Best-Fit" Distribution the Same for All Levels of a Factor?
6               1400         0
Randomness Analysis--How Random/Un-Autocorrelated is the Process?
6.1             1400        11
How Estimate Autocorrelation?
6.2             1400        25
How Attach Uncertainty to Autocorrelation Estimate?
6.3             1400        31
How Compute Confidence Interval for Autocorrelation Estimate?
6.4             1400        36
How Many Units for Prescribed Uncertainty in Autocorr. Est.?
6.5             1400        41
How Test to See if Eng. Mod. has Changed the Autocorr. Structure?
6.6             1400        46
Is Process Autocorrelation the Same for All Levels of a Factor?
7               1500         0
Stability Analysis--Is Process Same for All Levels of a Factor?
7.1             1500        12
Same in General?
7.2             1500        19
Same Typical Value (Location)?
7.3             1500        30
Same Precision (Variation)?
7.4             1500        49
Same Distribution?
7.5             1500        72
Same Autocorrelation?
7.6             1500        80
Same Linear Regression Parameters?
7.7             1500        91
Same Other?
8               1700         0
Uncertainty Analysis--How Attach Error Bars to an Estimator?
8.1             1700        17
Normal Distribution & Location Parameter mu
8.2             1700        24
Normal Distribution & Precision Parameter sigma
8.3             1700        32
Binomial Distribution and Proportion Parameter p
8.4             1700        41
Simulation/Random Number Generation
8.5             1700        88
Propagation of Error/Sensitivity Analysis
8.6             1700       103
Deriving Error Bars for Typical Value (Location) Estimates
8.7             1700       120
Deriving Error Bars for Precision (Variation) Estimates
8.8             1700       151
Deriving Error Bars for Distributional Shape Estimates
8.9             1700       192
Deriving Error Bars for Autocorrelation Estimates
8.10            1700       203
Deriving Error Bars for Linear Regression Parameter Estimates
8.11            1700       218
Deriving Error Bars for Other Estimates
8.12            1700       230
Displaying Error Bars on Plots
9               2000         0
Stat. Proc. Control--Is the Process Statistically "In Control"?
9.1             2000        10
Definition of Statistically "In Control"
9.2             2000        21
Is the process "in control" in general?
9.3             2000        27
Is the Typical Value (Location) Drifting?
9.4             2000        34
Is the Precision (Variation) Changing?
9.5             2000        42
Is the Number of Defectives Increasing?
10              2100         0
Reliability Analysis--What is Process/Product Lifetime?
10.1            2100         9
How Carry Out Classical (= 2-Parameter) Weibull Analysis?
10.1.1          2100        23
What is Assumed with the Classical Weibull Analysis?
10.1.2          2100        31
How Generate Classical Weibull Plot?
10.1.3          2100        37
How Estimate Weibull Shape Parameter beta?
10.1.4          2100        45
How Estimate Characteristic Life eta?
10.1.5          2100        53
How Estimate Lifetime Predictions?
10.1.6          2100        61
How Handle Multimode Failures?
10.1.7          2100        72
How Handle Suspended Test Items--Non-failures
10.1.8          2100        83
How Generate Best-Fit Weibull Density Function?
10.1.9          2100        90
How Superimpose Histogram with Best-Fit Weibull?
10.2            2100        95
How Carry Out Generalized (= 3-Parameter) Weibull Analysis?
10.2.1          2100       110
What is Assumed with the Extended Weibull Analysis?
10.2.2          2100       118
How Generate Extended Weibull Plot?
10.2.3          2100       126
How Estimate Weibull Shape Parameter gamma?
10.2.4          2100       134
How Estimate Characteristic Life eta?
10.2.5          2100       144
How Estimate Lower-Bound Location Parameter t0?
10.2.6          2100       154
How Estimate Lifetime Predictions?
10.2.7          2100       171
How Handle Multimode Failures?
10.2.8          2100       183
How Handle Suspended Test Items--Non-failures
10.2.9          2100       195
How Generate Best-Fit Weibull Density Function?
10.2.10         2100       207
How Superimpose Histogram with Best-Fit Weibull?
10.3            2100       212
How Carry Out Usual Alternate Non-Weibull Analysis?
10.4            2100       225
How Carry Out Classical Hazard Analysis?
11              2400         0
Warranty Analysis--How Determine a Warrenty Value for a Product?
11.1            2400         8
What are the Simplest Assumptions for a Warranty Analysis?
11.2            2400        17
What % of Units will Fail Warranty?
11.3            2400        27
How Choose a Warranty Value so at most p% of Units Fail?
12              2500         0
Proc. Capability Anal.--How "Good"/Capable is the Process?
12.1            2500        10
What are the Simplest Assumptions for a Process Capability Anal.?
12.2            2500        19
How Compute Capability Statistics?
12.3            2500        31
Are Capability Statistics the Same for All Levels of a Factor?
12.4            2500        43
How Generate a Histogram with Tolerance Limits?
12.5            2500        48
How Generate a Histogram with Superimposed Normal Density?
13              2600         0
Quality Analysis--How Can the Process be Improved?
13.1            2600         8
What are the 7 "Old" Tools for Quality Improvement?
13.1.1          2600        20
Check Sheet
13.1.2          2600        25
Pareto Diagram
13.1.3          2600        31
Ishikawa Diagram
13.1.4          2600        37
Histogram
13.1.5          2600        43
Stratified Histogram
13.1.6          2600        53
Scatter Plot
13.1.7          2600        59
Control Charts
13.2            2600        71
What are the 7 "New" Tools for Quality Improvement?
13.3            2600        89
What is Quality Function Deployment (QFD)?
14              2800         0
Time Series Analysis  --How Analyze Equi-spaced Data?
14.1            2800         7
How Examine a Univariate Time Series?
14.1.1          2800        20
How Determine Time-Domain Relatedness?
14.1.2          2800        27
How Estimate Parameters in Autoregressive Model?
14.1.3          2800        32
How Estimate Parameters in ARIMA Models?
14.1.4          2800        37
How Determine Frequency-Domain Relatedness?
14.1.5          2800        44
How Determine Low-Frequency-Domain Model?
14.1.6          2800        51
Constant Amplitude & Phase in Single-Cycle Sine Model?
14.1.7          2800        60
Is Autocorrelation the Same for All Levels of a Factor
14.1.8          2800        66
How Smooth/Filter the Data?
14.1.8.1        2800        83
Time Domain Smoothing (Low-Pass Filtering) (Least Squares)
14.1.8.2        2800        88
Time Domain Smoothing (Low-Pass Filtering) (Robust)
14.1.8.3        2800        93
Frequency Domain Smoothing (Low-Pass Filtering)
14.1.8.4        2800        98
Time Domain High-Pass Filtering (Least Squares)
14.1.8.5        2800       103
Time Domain High-Pass Filtering (Robust)
14.1.8.6        2800       108
Frequency Domain High-Pass Filtering
14.1.8.7        2800       113
Time Domain Band-Pass Filtering (Least Squares)
14.1.8.8        2800       118
Time Domain Band-Pass Filtering (Robust)
14.1.8.9        2800       123
Frequency Domain Band-Pass Filtering
14.2            2800       128
How Examine a Bivariate Time Series?
14.2.1          2800       136
How Determine Time-Domain Relatedness?
14.2.2          2800       148
How Estimate Parameters in Bi-regressive Model?
14.2.3          2800       153
How Determine Frequency-Domain Relatedness?
15              3000         0
Regression Analysis--How Determine the Model in Y = f(X1,X2,...)?
15.1            3000        10
Regression Assumptions
15.2            3000        19
General Regression Procedure
15.3            3000        27
Fit 1-Variable Model y = f(x)
15.3.1          3000        41
How Select the Proper Model?
15.3.2          3000        49
How Fit a Linear, Quadratic, or Polynomial Model?
15.3.3          3000        65
How Fit a General Model (Linear, Polynomial, Non-Linear, etc.)?
15.3.4          3000        71
How Fit a Rational Function Model?
15.3.5          3000        83
How Fit an Exponential Model?
15.3.6          3000        89
How Fit a Sinusoidal Model?
15.3.7          3000       102
How Do Residual Analysis?
15.3.8          3000       113
How Transform to Linear Model?
15.3.9          3000       119
How do Robust regression?
15.3.10         3000       124
How to Weighted Regression?
15.3.11         3000       129
How do Non-Least Squares Regression?
15.4            3000       134
Fit 2-Variable Model y = f(x1,x2)
15.4.1          3000       146
How Select the Proper Model?
15.4.2          3000       157
How Fit a 2-Variable Multilinear Model?
15.4.3          3000       166
How Fit a General Model (Multi-Linear, Non-Linear, etc.)?
15.4.4          3000       172
How Fit an Exponential Model in 2 Variables?
15.4.5          3000       178
How Do Residual Analysis?
15.4.6          3000       191
How do Robust Regression?
15.4.7          3000       196
How to Weighted Regression?
15.5            3000       201
Fit 3/4/5/...-Variable Model y = f(x1,x2,x3,...) (Multivariate Reg.)
15.5.1          3000       214
How Select the Proper Model?
15.5.2          3000       229
How Fit a k-Variable Multilinear Model?
16.5.3          3000       235
How Do All Possible Subsets Regression?
15.5.4          3000       240
How Fit a General Model (Multi-Linear, Non-Linear, etc.)?
15.5.5          3000       246
How Fit an Exponential Model in k Variables?
15.5.6          3000       251
How Do Residual Analysis?
15.5.7          3000       285
How do Robust Regression?
15.5.8          3000       290
How to Weighted Regression?
16              3300         0
Multifactor Analysis--What Factors Affect Process Yield?
17              3400         0
Multivariate Analysis--How Examine Multiple Responses?
17.1            3400        13
How Examine 2 Responses--Y1 and Y2?
17.2            3400        23
How Examine 3/4/5/... Responses--Y1, Y2, Y3,...?
17.3            3400        33
How Do Cross-Correlation?
17.4            3400        38
How Do Discrimination Analysis?
17.5            3400        43
How Do Cluster Analysis?
17.6            3400        48
How Do Cross-Tabulation?
17.7            3400        53
How Do Cannonical Analysis?
17.8            3400        58
How Do Principal Components Analysis?
18              3500         0
Interlab Analysis--Are All Labs Equally Good?
Your First      3500       158
Probability Calculation
~Built-in       3500       162
~Complex Arithmetic~
No menu it      3500       750
Please enter -1 to revert
Your First      3500       862
Probability Calculation
~Built-in       3500       866
~Complex Arithmetic~
No menu it      3500      1454
Please enter -1 to revert
 
 
 
 
 
 
 
 
 
 
 
 
 
 
 
 
 
 
 
 
 
 
 
 
 
 
 
 
 
 
 
 
 
 
 
 
 
 
 
 
 
 
 
 
 
 
 
 
 
 
 
 
 
 
 
 
 
 
 
 
 
 
 
 
 
 
 
 
 
 
 
 
 
 
 
 
 
 
 
 
 
 
 
 
 
 
 
 
 
 
 
 
 
 
 
 
 
----------  *INITIAL MENU FOR STATISTICS*  ------------
 
0
STATISTICS
   1. Assumptions       --What are Underlying Assumptions of a Proc.?
   2. Process Summariz. --How Summarize the Output from a Process?
   3. Location Analysis --What is the "Typical Value" of a Process?
   4. Variation Anal.   --What is the Precision of a Process?
   5. Distrib. Anal.    --What Distribution does the Process Follow?
   6. Randomness Anal.  --How Random/Un-autocorrelated is the Process?
   7. Stability Anal.   --Is Process Same for All Levels of a Factor?
   8. Uncertainty Anal. --How Attach Error Bars to an Estimator?
   9. Stat. Proc. Cont. --Is the Process Statistically "In Control"?
  10. Reliability Anal. --What is Process/Product Lifetime?
  11. Warranty Analysis --How Determine a Warrenty Value for a Product?
  12. Proc. Capab. Anal.--How "Good"/Capable is the Process?
  13. Quality Analysis  --How Can the Process be Improved?
  14. Time Series Anal. --How Analyze Equi-spaced Data?
  15. Regression Anal.  --How Determine the Model in Y = f(X1,X2,...)?
  16. Multifactor Anal. --What Factors Affect Process Yield?
  17. Multivariate Anal.--How Examine Multiple Responses?
  18. Interlab Analysis --Are All Labs Equally Good?
 
----------------------------------------------------------
 
 
 
 
 
 
 
 
 
 
 
 
 
 
 
 
 
 
 
 
 
 
 
 
 
 
 
 
 
 
 
 
 
 
 
 
 
 
 
 
 
 
 
 
 
 
 
 
 
 
 
 
 
 
 
 
 
 
 
 
 
 
 
 
 
 
 
 
 
 
 
 
 
 
 
 
----------  *ASSUMPTIONS*  -------------------------------
 
1.
Assumptions--What are Underlying Assumptions of a Process?
   1. What are the 4 Assumptions?
   2. What Gain is Achieved if Assumptions Hold?
   3. What Penalty is Incurred if Assumptions Fail?
   4. How Test the 4 Underlying Assumptions of a Process?
 
----------------------------------------------------------
 
1.1
What are the 4 Assumptions?
   1. Fixed "Typical Value" (Location)
   2. Fixed Precision (Variation)
   3. Fixed Distribution
   4. Randomness/Un-Autocorrelated
 
----------------------------------------------------------
 
1.2
What Gain is Achieved if Assumptions Hold?
   1. Predictability
   2. Valid Engineering/Scientific Conclusions
 
----------------------------------------------------------
 
1.3
What Penalty is Incurred if Assumptions Fail?
   1. Biased estimate of location
   2. Optimistically-small uncertainty estimates
   3. Poor predictive performance
   4. Invalid engineering/scientific conclusions
 
----------------------------------------------------------
 
1.4
How Test the 4 Underlying Assumptions of a Process?
      SKIP 25; READ LEW1.DAT Y
      4-PLOT Y
      SUMMARY Y
 
----------------------------------------------------------
 
 
 
 
 
 
 
 
 
 
 
 
 
 
 
 
 
 
 
 
 
 
 
 
 
 
 
 
 
 
 
 
 
 
 
 
 
 
 
 
 
 
 
 
 
 
 
 
 
 
 
 
 
 
 
 
 
----------  *PROCESS SUMMARIZATION*  -------------------------------
 
2
Process Summarization--How Summarize the Output from a Process?
   1. Number of Data Points = ...
   2. Estimate of "Typical Value" (Location) = ...
   3. Estimate of Precision (Variation) = ...
   4. Uncertainty in the "Typical Value" Estimate = ...
   5. Verbal Description of How the Uncertainty was Computed
   6. "Best-Fit" Distribution = ...
 
-----------------------------------------------------
 
2.1
Number of Data Points = ...
      SKIP 25; READ LEW1.DAT Y
      LET N = NUMBER Y
      SUMMARY Y
 
-----------------------------------------------------
 
2.2
Estimate of "Typical Value" (Location) = ...
      SKIP 25; READ LEW1.DAT Y
      LET XBAR = MEAN Y
      SUMMARY Y
 
-----------------------------------------------------
 
2.3
Estimate of Precision (Variation) = ...
      SKIP 25; READ LEW1.DAT Y
      LET S = STANDARD DEVIATION Y
      SUMMARY Y
 
-----------------------------------------------------
 
2.4
Uncertainty in the "Typical Value" Estimate = ...
      SKIP 25; READ LEW1.DAT Y
      LET XBAR = MEAN Y
      LET SDXBAR = STANDARD DEVIATION OF MEAN Y
      SUMMARY Y
 
-----------------------------------------------------
 
2.5
Verbal Description of How the Uncertainty was Computed
      SKIP 25; READ LEW1.DAT Y
      LET XBAR = MEAN Y
      LET SDXBAR = STANDARD DEVIATION OF THE MEAN Y
      LET UPPER = 2*SDXBAR
      WRITE "The estimate for the process mean is ^XBAR "
      WRITE "with an uncertainty of +- ^UPPER "
      WRITE "where the uncertainty (+- ^UPPER ) "
      WRITE "is 3 times the estimated standard "
      WRITE "deviation for the estimate. "
 
 
-----------------------------------------------------
 
2.6
"Best-Fit" Distribution = ...
      SKIP 25; READ LEW1.DAT Y
      NORMAL PROBABILITY PLOT Y
      TUKEY PPCC PLOT Y
      UNIFORM PROBABILITY PLOT Y
      ARCSIN PROBABILITY PLOT Y
 
----------------------------------------------------------
 
 
 
 
 
 
 
 
 
 
 
 
 
 
 
 
 
 
 
 
 
 
 
 
 
 
 
 
 
 
----------  *LOCATION ESTIMATION*  -------------------------------
 
3
Location Analysis--What is the "Typical Value" of a Process?
   1. How Estimate Location?
   2. How Attach Uncertainty to Location Estimate?
   3. How Compute Confidence Interval for Location Estimate?
   4. How Many Units for Prescribed Uncertainty in Loc. Est.?
   5. How Test to See if Eng. Mod. has Improved "Typical" Process Yield?
   6. Is "Typical" Process Yield the Same for All Levels of a Factor?
 
-----------------------------------------------------
 
3.1
How Estimate Location?
      SUMMARY Y
      LET M = MEAN Y
      LET M = MEDIAN Y
      LET M = MIDRANGE Y
      LET M = MIDMEAN Y
      LET P1 = .15
      LET P2 = .25
      LET M = TRIMMED MEAN Y
      LET P1 = .15
      LET P2 = .25
      LET M = WINSORIZED MEAN Y
 
-----------------------------------------------------
 
3.2
How Attach Uncertainty to Location Estimate?
      SUMMARY Y
      LET M = MEAN Y
      LET SDM = STANDARD DEVIATION OF THE MEAN Y
 
-----------------------------------------------------
 
3.3
How Compute Confidence Interval for Location Estimate?
      LET XBAR = MEAN Y
      LET SDXBAR = STNDARD DEVIATION OF THE MEAN Y
      LET N = NUMBER Y
      LET NU = N-1
      LET K1 = TPPF(.025,NU)
      LET K2 = TPPF(.025,NU)
      LET LOWER = XBAR+K1*SDXBAR
      LET UPPER = XBAR+K2*SDXBAR
 
-----------------------------------------------------
 
3.4
How Many Units for Prescribed Uncertainty in Loc. Est.?
 
-----------------------------------------------------
 
3.5
How Test to See if Eng. Mod. has Improved "Typical" Process Yield?
   1. Graphical Analysis
   2. Classical Analysis
   3. Non-parametric Analysis
 
-----------------------------------------------------
 
3.5.1
Graphical Analysis
      PLOT Y X
      BOX PLOT Y X
      MEAN PLOT Y X
      LET Y1 = EXTRACT Y X 1
      LET Y2 = EXTRACT Y X 2
      BIHISTOGRAM Y1 Y2
 
-----------------------------------------------------
 
3.5.2
Classical Analysis
      ANOVA Y X
      LET Y1 = EXTRACT Y X 1
      LET Y2 = EXTRACT Y X 2
      T TEST Y1 Y2
 
-----------------------------------------------------
 
3.5.3
Non-parametric Analysis
      LET Y = RANK Y
      ANOVA Y X
      LET Y1 = EXTRACT Y X 1
      LET Y2 = EXTRACT Y X 2
      BIHISTOGRAM Y1 Y2
      T TEST Y1 Y2
 
-----------------------------------------------------
 
3.6
Is "Typical" Process Yield the Same for All Levels of a Factor?
      PLOT Y X
      BOX PLOT Y X
      MEAN PLOT Y X
      MEDIAN PLOT Y X
      MIDMEAN PLOT Y X
      MIDRANGE PLOT Y X
      TRIMMED MEAN PLOT Y X
      WINDSORIZED MEAN PLOT Y X
      ANOVA Y X
      TABULATE MEANS Y X
 
-----------------------------------------------------
 
 
 
 
 
 
 
 
 
 
 
 
 
 
 
 
 
 
 
 
 
 
 
 
 
 
 
 
 
 
 
 
 
 
 
 
 
 
 
 
 
 
 
 
 
 
 
 
 
 
 
 
 
 
 
 
 
 
 
 
 
 
 
 
 
 
 
 
 
 
 
 
 
 
 
 
 
 
 
 
 
 
 
 
 
 
 
 
 
 
 
 
----------  *PRECISION ESTIMATION*  -------------------------------
 
4
Variation Analysis--What is the Precision of a Process?
   1. How Estimate Process Precision?
   2. How Attach Uncertainty to Precision Estimate?
   3. How Compute Confidence Interval for Precision Estimate?
   4. How Many Units for Prescribed Uncertainty in Prec. Est.?
   5. How Test to See if Eng. Mod. has Improved Process Precision?
   6. Is Process Precision the Same for All Levels of a Factor?
   7. How Transform to Fixed Process Precision?
 
-----------------------------------------------------
 
4.1
How Estimate Process Precision?
      SUMMARY Y
      LET S = STANDARD DEVIATION Y
      LET V = VARIANCE Y
      LET R = RANGE Y
 
-----------------------------------------------------
 
4.2
How Attach Uncertainty to Precision Estimate?
      NORMAL PROBABILITY PLOT Y (since validity assumes normality)
      LET S = STANDARD DEVIATION Y
      LET N = NUMBER Y
      LET NU = N-1
      LET K1 = CHSPPF(.025,NU)
      LET K2 = CHSPPF(.975,NU)
      LET LOWER = S*SQRT(NU/K2)
      LET UPPER = S*SQRT(NU/K1)
 
-----------------------------------------------------
 
4.3
How Compute Confidence Interval for Precision Estimate?
 
-----------------------------------------------------
 
4.4
How Many Units for Prescribed Uncertainty in Prec. Est.?
 
-----------------------------------------------------
 
4.5
How Test to See if Eng. Mod. has Improved Process Precision?
   1. Graphical
   2. Classical
   3. Non-parametric
 
-----------------------------------------------------
 
4.5.1
Graphical Analysis
      PLOT Y X
      BOX PLOT Y X
      STANDARD DEVIATION PLOT Y X
      LET Y1 = EXTRACT Y X 1
      LET Y2 = EXTRACT Y X 2
      BIHISTOGRAM Y1 Y2
 
-----------------------------------------------------
 
4.5.2
Classical Analysis
      ANOVA Y X
      LET N1 = NUMBER Y1
      LET N2 = NUMBER Y2
      LET NU1 = N1-1
      LET NU2 = N2-1
      LET LOWER = 0
      LET UPPER = FCDF(.95,NU1,NU2)
      LET V1 = VARIANCE Y1
      LET V2 = VARIANCE Y2
      LET FCALC = V1/V2
      LET FCALC = 1/FCALC IF FCALC < 1
 
-----------------------------------------------------
 
4.5.3
Non-parametric Analysis
      LET Y = RANK Y
      ...
 
-----------------------------------------------------
 
4.6
Is Process Precision the Same for All Levels of a Factor?
      PLOT Y X
      BOX PLOT Y X
      STANDARD DEVIATION PLOT Y X
      VARIANCE PLOT Y X
      RANGE PLOT Y X
      MINIMUM PLOT Y X
      MAXIMUM PLOT Y X
      RELATIVE SD PLOT Y X
      TAGUCHI ... PLOT Y X
         TAGUCHI SN PLOT Y X
         TAGUCHI SNL PLOT Y X
         TAGUCHI SNS PLOT Y X
         TAGUCHI SN2 PLOT Y X
      HOMOSCEDASTICITY PLOT Y X
      TABULATE STANDARD DEVIATIONS Y X
      TABULATE RANGES Y X
 
-----------------------------------------------------
 
4.7
How Transform to Fixed Process Precision?
      LET Y2 = LOG(Y)      frequently
      LET Y2 = SQRT(Y)     frequently
      BOX-COX HOMOSCEDASTICITY PLOT Y X
 
-----------------------------------------------------
 
 
 
 
 
 
 
 
 
 
 
 
 
 
 
 
 
 
 
 
 
 
 
 
 
 
 
 
 
 
 
 
 
 
 
 
 
 
 
 
 
 
 
 
 
 
 
 
 
 
 
 
 
 
 
 
 
 
 
 
 
 
 
 
 
 
 
 
 
 
 
 
 
 
 
 
 
 
 
 
 
 
 
 
----------  *DISTRIBUTION ANALYSIS*  -------------------------------
 
5
What Distribution does the Process Follow (Distrib. Analysis)?
   1. What is the General Distributional Shape?
   2. What Distributional Family Best Fits the Data?
   3. What Member of a Distributional Family Best Fits the Data?
   4. What Specific Distribution Best Fits the Data?
   5. How Attach Uncertainty to "Best-Fit" Dist. Estimate?
   6. How Compute Confidence Interval for "Best-Fit" Dist. Estimate?
   7. How Many Units for Prescribed Uncertainty in "Best=Fit" Dist.?
   8. How Test to See if Eng. Mod. has Changed "Best-Fit" Distribution?
   9. Is "Best-Fit" Distribution the Same for All Levels of a Factor?
 
-----------------------------------------------------
 
5.1
What is the General Distributional Shape?
      ... HISTOGRAM Y
         CUMULATIVE HISTOGRAM Y
         RELATIVE HISTOGRAM Y
         CUMULATIVE RELATIVE HISTOGRAM Y
      STEM-AND-LEAF PLOT Y
      BOX PLOT Y
      ROOTOGRAM Y
      PERCENT POINT PLOT Y
      PIE CHART Y
      SYMMETRY PLOT Y
      TABULATE Y
 
-----------------------------------------------------
 
5.2
What Distributional Family Best Fits the Data?
      CHI-SQUARED PPCC PLOT Y
      EXTREME VALUE TYPE 2 PPCC PLOT Y
      FATIGUE LIFE PPCC PLOT Y
      GAMMA PPCC PLOT Y
      GEOMETRIC PPCC PLOT Y
      INVERSE GAUUSIAN PPCC PLOT Y
      PARETO PPCC PLOT Y
      POISSON PPCC PLOT Y
      PPCC PLOT Y    (defaults to TUKEY PPCC PLOT)
      RECIPROCAL INVERSE GAUUSIAN PPCC PLOT Y
      T PPCC PLOT Y
      TUKEY PPCC PLOT Y
      WEIBULL PPCC PLOT Y
 
-----------------------------------------------------
 
5.3
What Member of a Distributional Family Best Fits the Data?
      CHI-SQUARED PPCC PLOT Y
      EXTREME VALUE TYPE 2 PPCC PLOT Y
      FATIGUE LIFE PPCC PLOT Y
      GAMMA PPCC PLOT Y
      GEOMETRIC PPCC PLOT Y
      INVERSE GAUUSIAN PPCC PLOT Y
      PARETO PPCC PLOT Y
      POISSON PPCC PLOT Y
      PPCC PLOT Y    (defaults to TUKEY PPCC PLOT)
      RECIPROCAL INVERSE GAUUSIAN PPCC PLOT Y
      T PPCC PLOT Y
      TUKEY PPCC PLOT Y
      WEIBULL PPCC PLOT Y
 
-----------------------------------------------------
 
5.4
What Specific Distribution Best Fits the Data?
   1.  Is the Data Normal?
   2.  If Non-normal, how Transform to Normal?
   3.  Is the data Weibull?
   4.  Is the Data Uniform?
   5.  Is the Data Some Other Distribution?
 
-----------------------------------------------------
 
5.4.1
Is the Data Normal?
      NORMAL PLOT Y
      NORMAL PROBABILITY PLOT Y
      SUMMARY Y
      TUKEY PPCC PLOT Y
      ...LET C = NORMAL PPCC Y
 
-----------------------------------------------------
 
5.4.2
If Non-normal, how Transform to Normal?
      BOX-COX NORMALITY PLOT Y
 
-----------------------------------------------------
 
5.4.3
Is the Data Weibull?
      WEIBULL PLOT Y
      WEIBULL PROBABILITY PLOT Y
      WEIBULL PPCC PLOT Y
 
-----------------------------------------------------
 
5.4.4
Is the Data Uniform?
       UNIFORM PROBABILITY PLOT Y
       TUKEY PPCC PLOT Y
 
-----------------------------------------------------
 
5.4.5
Is the Data Some Other Distribution?
      NORMAL PROBABILITY PLOT Y
      UNIFORM PROBABILITY PLOT Y
      LOGISTIC PROBABILITY PLOT Y
      DOUBLE EXPO PROB PLOT Y
      CAUCHY PROBABILITY PLOT Y
      SEMI-CIRULAR PROB PLOT Y
      TRIANGULAR PROB PLOT Y
      HALFNORMAL PROB PLOT Y
      LOGNORMAL PROBABILITY PLOT Y
      EXPONENTIAL PROBABILITY PLOT Y
      EXTR VALUE TYPE 1 PROB PLOT Y
 
      LET P = 2
      LET Q = 1
      BETA PROBABILITY PLOT Y
      LET N = 20
      LET P = .4
      BINOMIAL PROBABILITY PLOT Y
      LET NU = 15
      CHI-SQUARED PROB PLOT Y
      LET GAMMA = 2
      EXTR VALUE TYPE 2 PROB PLOT Y
      LET NU1 = 7
      LET NU2 = 3
      F PROBABILITY PLOT Y
      LET GAMMA = 4
      GAMMA PROBABILITY PLOT Y
      LET P = .3
      GEOMETRIC PROB PLOT Y
      LET P = .3
      LET K = 5
      NEGATIVE BINO PROB PLOT Y
      LET GAMMA = 3
      PARETO PROBABILITY PLOT Y
      LET LAMBDA = 45
      POISSON PROBABILITY PLOT Y
      LET NU = 3
      T PROBABILITY PLOT Y
      LET LAMBDA = .14
      TUKEY LAMBDA PROB PLOT Y
      LET GAMMA = 3
      WEIBULL PROBABILITY PLOT Y
 
-----------------------------------------------------
 
5.5
How Attach Uncertainty to "Best-Fit" Dist. Estimate?
 
-----------------------------------------------------
 
5.6
How Compute Confidence Interval for "Best-Fit" Dist. Estimate?
 
-----------------------------------------------------
 
5.7
How Many Units for Prescribed Uncertainty in "Best=Fit" Dist.?
 
-----------------------------------------------------
 
5.8
How Test to See if Eng. Mod. has Changed "Best-Fit" Distribution?
      BOX PLOT Y X
      LET Y1 = Y
      LET Y2 = Y
      LET Y1 = Y SUBSET X 1
      LET Y2 = Y SUBSET X 2
      BIHISTOGRAM Y1 Y2
      QUANTILE QUANTILE PLOT Y1 Y2
 
-----------------------------------------------------
 
5.9
Is "Best-Fit" Distribution the Same for All Levels of a Factor?
      BOX PLOT Y X
      SKEWNESS PLOT Y X
      KURTOSIS PLOT Y X
      MULTIPLOT 4 4
      LET DX = DISTINCT X
      LET KMAX = NUMBER DMAX
      LOOP FOR K = 1 1 KMAX
      NORMAL PROBABILITY PLOT SUBSET TAG K
      END OF LOOP
 
-----------------------------------------------------
 
 
 
 
----------  *RANDOMNESS ANALYSIS*  -------------------------------
 
6
Randomness Analysis--How Random/Un-Autocorrelated is the Process?
   1. How Estimate Autocorrelation?
   2. How Attach Uncertainty to Autocorrelation Estimate?
   3. How Compute Confidence Interval for Autocorrelation Estimate?
   4. How Many Units for Prescribed Uncertainty in Autocorr. Est.?
   5. How Test to See if Eng. Mod. has Changed the Autocorr. Structure?
   6. Is Process Autocorrelation the Same for All Levels of a Factor?
 
-----------------------------------------------------
 
6.1
How Estimate Autocorrelation?
      LET C = AUTOCORRELATION Y
      LAG ... PLOT Y
         LAG 1 PLOT Y  (or simply LAG PLOT Y)
         LAG 2 PLOT Y
         LAG 3 PLOT Y
         LAG <any positive integer> PLOT Y
      AUTOCORRELATION PLOT Y
      LET Y = RANK Y
      AUTOCORRELATION PLOT Y
 
-----------------------------------------------------
 
6.2
How Attach Uncertainty to Autocorrelation Estimate?
      AUTOCORRELATION STATISTIC PLOT Y X
 
-----------------------------------------------------
 
6.3
How Compute Confidence Interval for Autocorrelation Estimate?
 
-----------------------------------------------------
 
6.4
How Many Units for Prescribed Uncertainty in Autocorr. Est.?
 
-----------------------------------------------------
 
6.5
How Test to See if Eng. Mod. has Changed the Autocorr. Structure?
 
-----------------------------------------------------
 
6.6
Is Process Autocorrelation the Same for All Levels of a Factor?
 
-----------------------------------------------------
 
 
 
 
 
 
 
 
 
 
 
 
 
 
 
 
 
 
 
 
 
 
 
 
 
 
 
 
 
 
 
 
 
 
 
 
 
 
 
 
 
 
 
 
 
 
 
 
----------  *STABILITY ANALYSIS*  -------------------------------
 
7
Stability Analysis--Is Process Same for All Levels of a Factor?
   1. Same in General?
   2. Same Typical Value (Location)?
   3. Same Precision (Variation)?
   4. Same Distribution?
   5. Same Autocorrelation?
   6. Same Linear Regression Parameters?
   7. Same Other?
 
-----------------------------------------------------
 
7.1
Same in General?
      BOX PLOT Y X
      I PLOT Y X
 
-----------------------------------------------------
 
7.2
Same Typical Value (Location)?
      MEAN PLOT Y X
      MEDIAN PLOT Y X
      MIDMEAN PLOT Y X
      MIDRANGE PLOT Y X
      TRIMMED MEAN PLOT Y X
      WINDSORIZED MEAN PLOT Y X
 
-----------------------------------------------------
 
7.3
Same Precision (Variation)?
      MINIMUM PLOT Y X
      MAXIMUM PLOT Y X
      RANGE PLOT Y X
      STANDARD DEVIATION PLOT Y X
      VARIANCE PLOT Y X
      RELATIVE SD PLOT Y X
      STAN DEVI OF THE MEAN PLOT Y X
      VARIANCE OF THE MEAN PLOT Y X
      TAGUCHI ... PLOT Y X
         TAGUCHI SN PLOT Y X
         TAGUCHI SNL PLOT Y X
         TAGUCHI SNS PLOT Y X
         TAGUCHI SN2 PLOT Y X
         HOMOSCEDASTICITY PLOT Y X
 
-----------------------------------------------------
 
7.4
Same Distribution?
      ... QUARTILE PLOT Y X
          LOWER QUARTILE PLOT Y X
          UPPER QUARTILE PLOT Y X
      ... HINGE PLOT Y X
          LOWER HINGE PLOT Y X
          UPPER HINGE PLOT Y X
      ... DECILE PLOT Y X
          FIRST DECILE PLOT Y X
          SECOND DECILE PLOT Y X
          THIRD DECILE PLOT Y X
          FOURTH DECILE PLOT Y X
          FIFTH DECILE PLOT Y X
          SIXTH DECILE PLOT Y X
          SEVENTH DECILE PLOT Y X
          EIGHTH DECILE PLOT Y X
          NINTH DECILE PLOT Y X
      SKEWNESS PLOT Y X
      KURTOSIS PLOT Y X
 
-----------------------------------------------------
 
7.5
Same Autocorrelation?
      AUTOCORRELATION STATISTIC PLOT Y X
      AUTOCOVARIANCE PLOT Y X
      RANK COVARIANCE PLOT Y X
 
-----------------------------------------------------
 
7.6
Same Linear Regression Parameters?
      LINEAR ... PLOT Y X
         LINEAR CORRELATION PLOT Y X
 
         LINEAR INTERCEPT PLOT Y X
         LINEAR SLOPE PLOT Y X
         LINEAR RESSD PLOT Y X
 
-----------------------------------------------------
 
7.7
Same Other?
      COUNTS PLOT Y X
      SUM PLOT Y X
      PRODUCT PLOT Y X
      PROPORTION PLOT Y X
      SINE ... PLOT Y X
         SINE AMPLITUDE PLOT Y X
         SINE FREQUENCY PLOT Y X
 
-----------------------------------------------------
 
 
 
 
 
 
 
 
 
 
 
 
 
 
 
 
 
 
 
 
 
 
 
 
 
 
 
 
 
 
 
 
 
 
 
 
 
 
 
 
 
 
 
 
 
 
 
 
 
 
 
 
 
 
 
 
 
 
 
 
 
 
 
 
 
 
 
 
 
 
 
 
 
 
 
 
 
 
 
 
 
 
 
 
 
 
 
 
 
 
 
 
 
 
 
 
----------  *UNCERTAINTY ANALYSIS*  -------------------------------
 
8
Uncertainty Analysis--How Attach Error Bars to an Estimator?
   1. Normal Distribution & Location Parameter mu
   2. Normal Distribution & Precision Parameter sigma
   3. Binomial Distribution and Proportion Parameter p
   4. Simulation/Random Number Generation
   5. Propagation of Error/Sensitivity Analysis
   6. Deriving Error Bars for Typical Value (Location) Estimates
   7. Deriving Error Bars for Precision (Variation) Estimates
   8. Deriving Error Bars for Distributional Shape Estimates
   9. Deriving Error Bars for Autocorrelation Estimates
  10. Deriving Error Bars for Linear Regression Parameter Estimates
  11. Deriving Error Bars for Other Estimates
  12. Displaying Error Bars on Plots
 
-----------------------------------------------------
 
8.1
Normal Distribution & Location Parameter mu
      LET XBAR = MEAN Y
      LET SDXBAR = STANDARD DEVIATION OF THE MEAN Y
 
-----------------------------------------------------
 
8.2
Normal Distribution & Precision Parameter sigma
      LET S = STANDARD DEVIATION Y
      LET N = NUMBER Y
      LET SDS = (1/SQRT(2))*S/SQRT(N)
 
-----------------------------------------------------
 
8.3
Binomial Distribution and Proportion Parameter p
      LET N = NUMBER Y
      LET NSUCCESS = NUMBER Y SUBSET Y 1
      LET P = NSUCCESS/N
      LET SDP = SQRT(P*(1-P)/N)
 
-----------------------------------------------------
 
8.4
Simulation/Random Number Generation
       LET Y = NORMAL RANDOM NUMBERS FOR I = 1 1 20
       LET Y = UNIFORM RANDOM NUMBERS FOR I = 1 1 20
       LET Y = LOGISTIC RANDOM NUMBERS FOR I = 1 1 20
       LET Y = DOUBLE EXPO PROB PLOT Y
       LET Y = CAUCHY RANDOM NUMBERS FOR I = 1 1 20
       LET Y = SEMI-CIRULAR PROB PLOT Y
       LET Y = TRIANGULAR PROB PLOT Y
       LET Y = HALFNORMAL PROB PLOT Y
       LET Y = LOGNORMAL RANDOM NUMBERS FOR I = 1 1 20
       LET Y = EXPONENTIAL RANDOM NUMBERS FOR I = 1 1 20
       LET Y = EXTR VALUE TYPE 1 PROB PLOT Y
 
       LET P = 2
       LET Q = 1
       LET Y = BETA RANDOM NUMBERS FOR I = 1 1 20
       LET N = 20
       LET P = .4
       LET Y = BINOMIAL RANDOM NUMBERS FOR I = 1 1 20
       LET NU = 15
       LET Y = CHI-SQUARED RANDOM NUMBERS FOR I = 1 1 20
       LET GAMMA = 2
       LET Y = EXTR VALUE TYPE 2 RANDOM NUMBERS FOR I = 1 1 20
       LET NU1 = 7
       LET NU2 = 3
       LET Y = F RANDOM NUMBERS FOR I = 1 1 20
       LET GAMMA = 4
       LET Y = GAMMA RANDOM NUMBERS FOR I = 1 1 20
       LET P = .3
       LET Y = GEOMETRIC RANDOM NUMBERS FOR I = 1 1 20
       LET P = .3
       LET K = 5
       LET Y = NEGATIVE BINO RANDOM NUMBERS FOR I = 1 1 20
       LET GAMMA = 3
       LET Y = PARETO RANDOM NUMBERS FOR I = 1 1 20
       LET LAMBDA = 45
       LET Y = POISSON RANDOM NUMBERS FOR I = 1 1 20
       LET NU = 3
       LET Y = T RANDOM NUMBERS FOR I = 1 1 20
       LET LAMBDA = .14
       LET Y = TUKEY LAMBDA RANDOM NUMBERS FOR I = 1 1 20
       LET GAMMA = 3
       LET Y = WEIBULL RANDOM NUMBERS FOR I = 1 1 20
 
-----------------------------------------------------
 
8.5
Propagation of Error/Sensitivity Analysis
      LET X0 = 5
      LET Z = NORMAL RANDOM NUMBERS FOR I = 1 1 20
      LET X = X0+.1*Z
      LET Y = X**2
      MULTIPLOT 2 2
      X3LABEL AUTOMATIC
      PLOT X
      PLOT Y
      HISTOGRAM X
      HISTOGRAM Y
 
-----------------------------------------------------
 
8.6
Deriving Error Bars for Typical Value (Location) Estimates
      BOOTSTRAP MEAN PLOT Y
      BOOTSTRAP MEDIAN PLOT Y
      BOOTSTRAP MIDMEAN PLOT Y
      BOOTSTRAP MIDRANGE PLOT Y
      BOOTSTRAP TRIMMED MEAN PLOT Y
      BOOTSTRAP WINDSORIZED MEAN PLOT Y
      JACKNIFE  MEAN PLOT Y
      JACKNIFE  MEDIAN PLOT Y
      JACKNIFE  MIDMEAN PLOT Y
      JACKNIFE  MIDRANGE PLOT Y
      JACKNIFE  TRIMMED MEAN PLOT Y
      JACKNIFE  WINDSORIZED MEAN PLOT Y
 
-----------------------------------------------------
 
8.7
Deriving Error Bars for Precision (Variation) Estimates
      BOOTSTRAP MINIMUM PLOT Y
      BOOTSTRAP MAXIMUM PLOT Y
      BOOTSTRAP RANGE PLOT Y
      BOOTSTRAP STANDARD DEVIATION PLOT Y
      BOOTSTRAP VARIANCE PLOT Y
      BOOTSTRAP RELATIVE SD PLOT Y
      BOOTSTRAP STAN DEVI OF THE MEAN PLOT Y
      BOOTSTRAP VARIANCE OF THE MEAN PLOT Y
      BOOTSTRAP TAGUCHI ... PLOT Y
         BOOTSTRAP TAGUCHI SN PLOT Y
         BOOTSTRAP TAGUCHI SNL PLOT Y
         BOOTSTRAP TAGUCHI SNS PLOT Y
         BOOTSTRAP TAGUCHI SN2 PLOT Y
      JACKNIFE  MINIMUM PLOT Y
      JACKNIFE  MAXIMUM PLOT Y
      JACKNIFE  RANGE PLOT Y
      JACKNIFE  STANDARD DEVIATION PLOT Y
      JACKNIFE  VARIANCE PLOT Y
      JACKNIFE  RELATIVE SD PLOT Y
      JACKNIFE  STAN DEVI OF THE MEAN PLOT Y
      JACKNIFE  VARIANCE OF THE MEAN PLOT Y
      JACKNIFE  TAGUCHI ... PLOT Y
         JACKNIFE  TAGUCHI SN PLOT Y
         JACKNIFE  TAGUCHI SNL PLOT Y
         JACKNIFE  TAGUCHI SNS PLOT Y
         JACKNIFE  TAGUCHI SN2 PLOT Y
 
-----------------------------------------------------
 
8.8
Deriving Error Bars for Distributional Shape Estimates
      BOOTSTRAP ... QUARTILE PLOT Y
         BOOTSTRAP LOWER QUARTILE PLOT Y
         BOOTSTRAP UPPER QUARTILE PLOT Y
      BOOTSTRAP ... HINGE PLOT Y
         BOOTSTRAP LOWER HINGE PLOT Y
         BOOTSTRAP UPPER HINGE PLOT Y
      BOOTSTRAP ... DECILE PLOT Y
         BOOTSTRAP FIRST DECILE PLOT Y
         BOOTSTRAP SECOND DECILE PLOT Y
         BOOTSTRAP THIRD DECILE PLOT Y
         BOOTSTRAP FOURTH DECILE PLOT Y
         BOOTSTRAP FIFTH DECILE PLOT Y
         BOOTSTRAP SIXTH DECILE PLOT Y
         BOOTSTRAP SEVENTH DECILE PLOT Y
         BOOTSTRAP EIGHTH DECILE PLOT Y
         BOOTSTRAP NINTH DECILE PLOT Y
      BOOTSTRAP SKEWNESS PLOT Y
      BOOTSTRAP KURTOSIS PLOT Y
      JACKNIFE  ... QUARTILE PLOT Y
         JACKNIFE  LOWER QUARTILE PLOT Y
         JACKNIFE  UPPER QUARTILE PLOT Y
      JACKNIFE  ... HINGE PLOT Y
         JACKNIFE  LOWER HINGE PLOT Y
         JACKNIFE  UPPER HINGE PLOT Y
      JACKNIFE  ... DECILE PLOT Y
         JACKNIFE  FIRST DECILE PLOT Y
         JACKNIFE  SECOND DECILE PLOT Y
         JACKNIFE  THIRD DECILE PLOT Y
         JACKNIFE  FOURTH DECILE PLOT Y
         JACKNIFE  FIFTH DECILE PLOT Y
         JACKNIFE  SIXTH DECILE PLOT Y
         JACKNIFE  SEVENTH DECILE PLOT Y
         JACKNIFE  EIGHTH DECILE PLOT Y
         JACKNIFE  NINTH DECILE PLOT Y
      JACKNIFE  SKEWNESS PLOT Y
      JACKNIFE  KURTOSIS PLOT Y
 
-----------------------------------------------------
 
8.9
Deriving Error Bars for Autocorrelation Estimates
      BOOTSTRAP AUTOCORR PLOT Y
      BOOTSTRAP AUTOCOV PLOT Y
      BOOTSTRAP RANK COVAR PLOT Y
      JACKNIFE  AUTOCORR PLOT Y
      JACKNIFE  AUTOCOV PLOT Y
      JACKNIFE  RANK COVAR PLOT Y
 
-----------------------------------------------------
 
8.10
Deriving Error Bars for Linear Regression Parameter Estimates
      BOOTSTRAP LINEAR ... PLOT Y X
         BOOTSTRAP LINEAR CORRELATION PLOT Y X
         BOOTSTRAP LINEAR INTERCEPT PLOT Y X
         BOOTSTRAP LINEAR SLOPE PLOT Y X
         BOOTSTRAP LINEAR RESSD PLOT Y X
      JACKNIFE  LINEAR ... PLOT Y X
         JACKNIFE  LINEAR CORRELATION PLOT Y X
         JACKNIFE  LINEAR INTERCEPT PLOT Y X
         JACKNIFE  LINEAR SLOPE PLOT Y X
         JACKNIFE  LINEAR RESSD PLOT Y X
 
-----------------------------------------------------
 
8.11
Deriving Error Bars for Other Estimates
      BOOTSTRAP COUNTS PLOT Y
      BOOTSTRAP SUM PLOT Y
      BOOTSTRAP PRODUCT PLOT Y
      BOOTSTRAP PROPORTION PLOT Y
      BOOTSTRAP SINE ... PLOT Y
         BOOTSTRAP SINE AMPLITUDE PLOT Y
         BOOTSTRAP SINE FREQUENCY PLOT Y
 
-----------------------------------------------------
 
8.12
Displaying Error Bars on Plots
      ERROR BAR Y X
      ERROR BAR Y X ...
 
-----------------------------------------------------
 
 
 
 
 
 
 
 
 
 
 
 
 
 
 
 
 
 
 
 
 
 
 
 
 
 
 
 
 
 
 
 
 
 
 
 
 
 
 
 
 
 
 
 
 
 
 
 
 
 
 
 
 
 
 
 
 
 
 
 
 
 
----------  *STATISTICAL PROCESS CONTROL*  ----------------------
 
9
Stat. Proc. Control--Is the Process Statistically "In Control"?
   1. Definition of Statistically "In Control"
   2. Is the Process "In Control" In General?
   3. Is the Typical Value (Location) Drifting?
   4. Is the Precision (Variation) Changing?
   5. Is the Number of Defectives Increasing?
 
----------------------------------------------------------
 
9.1
Definition of Statistically "In Control"
   A process is statistically "in control" if
   the output from the process "behaves like"
      1. random drawings
      2. from a fixed distribution
      3. with fixed location
      4. and fixed precision.
 
----------------------------------------------------------
 
9.2
Is the process "in control" in general?
      4-PLOT Y
 
----------------------------------------------------------
 
9.3
Is the Typical Value (Location) Drifting?
      XBAR CONTROL CHART Y X
      4-PLOT Y
 
----------------------------------------------------------
 
9.4
Is the Precision (Variation) Changing?
      R CONTROL CHART Y X
      STANDARD DEVIATION CONTROL CHART Y X
      4-PLOT Y
 
----------------------------------------------------------
 
9.5
Is the Number of Defectives Increasing?
      C CONTROL CHART Y X
      P CONTROL CHART Y X
      NP CONTROL CHART Y X
      U CONTROL CHART Y X
 
----------------------------------------------------------
 
 
 
 
 
 
 
 
 
 
 
 
 
 
 
 
 
 
 
 
 
 
 
 
 
 
 
 
 
 
 
 
 
 
 
 
 
 
 
 
 
 
 
 
 
 
 
 
----------  *RELIABILITY ANALYSIS*  ----------------------
 
10
Reliability Analysis--What is Process/Product Lifetime?
   1. How Carry Out Classical   (= 2-Parameter) Weibull Analysis?
   2. How Carry Out Generalized (= 3-Parameter) Weibull Analysis?
   3. How Carry Out Usual Alternate Non-Weibull Analysis?
   4. How Carry Out Classical Hazard Analysis?
 
----------------------------------------------------------
 
10.1
How Carry Out Classical (= 2-Parameter) Weibull Analysis?
   1. What is Assumed with the Classical Weibull Analysis?
   2. How Generate Classical Weibull Plot?
   3. How Estimate Weibull Shape Parameter beta?
   4. How Estimate Characteristic Life eta?
   5. How Estimate Lifetime Predictions?
   6. How Handle Multimode Failures?
   7. How Handle Suspended Test Items--Non-failures
   8. How Generate Best-Fit Weibull Density Function?
   9. How Superimpose Histogram with Best-Fit Weibull?
 
----------------------------------------------------------
 
10.1.1
What is Assumed with the Classical Weibull Analysis?
      The scale parameter eta is allowed to float;
      the shape parameter beta is allowed to float;
      but the lower-bound location parameter t0 is set to 0.
 
----------------------------------------------------------
 
10.1.2
How Generate Classical Weibull Plot?
      WEIBULL PLOT Y
 
----------------------------------------------------------
 
10.1.3
How Estimate Weibull Shape Parameter beta?
      WEIBULL PLOT Y
      WRITE BETA SDBETA
      STATUS
 
----------------------------------------------------------
 
10.1.4
How Estimate Characteristic Life eta?
      WEIBULL PLOT Y
      WRITE ETA SDETA
      STATUS
 
----------------------------------------------------------
 
10.1.5
How Estimate Lifetime Predictions?
      WEIBULL PLOT Y
      WRITE B0P5 B1 B5 B10 B90
      STATUS
 
----------------------------------------------------------
 
10.1.6
How Handle Multimode Failures?
      WEIBULL PLOT Y TAG
      where TAG is a user-defined variable containing
      1's if the failure time in Y was from the
      failure mode of interest, and
      0's if the failure time in Y was from a
      non-interesting failure mode.
 
----------------------------------------------------------
 
10.1.7
How Handle Suspended Test Items--Non-failures
      WEIBULL PLOT Y TAG
      where TAG is a user-defined variable containing
      1's if the failure time in Y was from a
      failed item, and
      0's if the (bogus; e.g., 999) time in Y was from a
      non-failed (that is, suspended) item.
 
----------------------------------------------------------
 
10.1.8
How Generate Best-Fit Weibull Density Function?
      WEIBULL PLOT Y
      PLOT ETA*WEIPDF(X,BETA) FOR X = .1 .1 20
 
----------------------------------------------------------
 
10.1.9
How Superimpose Histogram with Best-Fit Weibull?
 
----------------------------------------------------------
 
10.2
How Carry Out Generalized (= 3-Parameter) Weibull Analysis?
   1. What is Assumed with the Extended Weibull Analysis?
   2. How Generate Extended Weibull Plot?
   3. How Estimate Weibull Shape Parameter gamma?
   4. How Estimate Characteristic Life eta?
   5. How Estimate Lower-Bound Location Parameter t0?
   6. How Estimate Lifetime Predictions?
   7. How Handle Multimode Failures?
   8. How Handle Suspended Test Items--Non-failures
   9. How Generate Best-Fit Weibull Density Function?
  10. How Superimpose Histogram with Best-Fit Weibull?
 
----------------------------------------------------------
 
10.2.1
What is Assumed with the Extended Weibull Analysis?
      The scale parameter eta is allowed to float;
      the shape parameter gamma is allowed to float;
      the lower-bound location parameter t0 is allowed to float.
 
----------------------------------------------------------
 
10.2.2
How Generate Extended Weibull Plot?
      WEIBULL PPCC PLOT Y
      LET GAMMA = <best value from WEIBULL PPCC PLOT Y>
      WEIBULL PROBABILITY PLOT Y
 
----------------------------------------------------------
 
10.2.3
How Estimate Weibull Shape Parameter gamma?
      WEIBULL PPCC PLOT Y
      The shape parameter gamma will be the horizontal axis
      value in which the PPCC curve maximizes itself.
 
----------------------------------------------------------
 
10.2.4
How Estimate Characteristic Life eta?
      WEIBULL PPCC PLOT Y
      LET GAMMA = <best value from WEIBULL PPCC PLOT Y>
      WEIBULL PROBABILITY PLOT Y
      LINEAR FIT YPLOT XPLOT
      LET ETA = A1
 
----------------------------------------------------------
 
10.2.5
How Estimate Lower-Bound Location Parameter t0?
      WEIBULL PPCC PLOT Y
      LET GAMMA = <best value from WEIBULL PPCC PLOT Y>
      WEIBULL PROBABILITY PLOT Y
      LINEAR FIT YPLOT XPLOT
      LET T0 = A0
 
----------------------------------------------------------
 
10.2.6
How Estimate Lifetime Predictions?
      WEIBULL PPCC PLOT Y
      LET GAMMA = <best value from WEIBULL PPCC PLOT Y>
      WEIBULL PROBABILITY PLOT Y
      LINEAR FIT YPLOT XPLOT
      LET ETA = A1
      LET T0 = A0
      LET B0P5 = T0+ETA*WEIPPF(.005,GAMMA)
      LET B1 = T0+ETA*WEIPPF(.01,GAMMA)
      LET B5 = T0+ETA*WEIPPF(.05,GAMMA)
      LET B10 = T0+ETA*WEIPPF(.05,GAMMA)
      LET B90 = T0+ETA*WEIPPF(.90,GAMMA)
      WRITE B0P5 B1 B5 B10 B90
 
----------------------------------------------------------
 
10.2.7
How Handle Multimode Failures?
      WEIBULL PPCC PLOT Y
      LET GAMMA = <best value from WEIBULL PPCC PLOT Y>
      WEIBULL PROBABILITY PLOT Y
      LINEAR FIT YPLOT XPLOT
      LET T0 = A0
      LET Y2 = Y-T0
      WEIBULL PLOT Y2 TAG
 
----------------------------------------------------------
 
10.2.8
How Handle Suspended Test Items--Non-failures
      WEIBULL PPCC PLOT Y
      LET GAMMA = <best value from WEIBULL PPCC PLOT Y>
      WEIBULL PROBABILITY PLOT Y
      LINEAR FIT YPLOT XPLOT
      LET T0 = A0
      LET Y2 = Y-T0
      WEIBULL PLOT Y2 TAG
 
----------------------------------------------------------
 
10.2.9
How Generate Best-Fit Weibull Density Function?
      WEIBULL PPCC PLOT Y
      LET GAMMA = <best value from WEIBULL PPCC PLOT Y>
      WEIBULL PROBABILITY PLOT Y
      LINEAR FIT YPLOT XPLOT
      LET ETA = A1
      LET T0 = A0
      PLOT T0+ETA*WEIPDF(X,GAMMA) FOR X = .1 .1 20
 
----------------------------------------------------------
 
10.2.10
How Superimpose Histogram with Best-Fit Weibull?
 
----------------------------------------------------------
 
10.3
How Carry Out Usual Alternate Non-Weibull Analysis?
      NORMAL PROBABILITY PLOT Y
      LOGNORMAL PROBABILITY PLOT Y
      HALFNORMAL PROBABILITY PLOT Y
      EXTREME VALUE TYPE 1 PROABILITY PLOT Y
      EXTREME VALUE TYPE 2 PPCC PLOT Y
      INVERSE GAUSSIAN PPCC PLOT Y
      RECIPROCAL INVERSE GAUSSIAN PPCC PLOT Y
      FATIGUE LIFE PPCC PLOT Y
 
----------------------------------------------------------
 
10.4
How Carry Out Classical Hazard Analysis?
 
----------------------------------------------------------
 
 
 
 
 
 
 
 
 
 
 
 
 
 
 
 
 
 
 
 
 
 
 
 
 
 
 
 
 
 
 
 
 
 
 
 
 
 
 
 
 
 
 
 
 
 
 
 
 
 
 
 
 
 
 
 
 
 
 
 
 
 
 
 
 
 
 
 
 
----------  *WARRANTY ANALYSIS*  ----------------------
 
11
Warranty Analysis--How Determine a Warrenty Value for a Product?
   1. What are the Simplest Assumptions for a Warranty Analysis?
   2. What % of Units will Fail Warranty?
   3. How Choose a Warranty Value so at most p% of Units Fail?
 
----------------------------------------------------------
 
11.1
What are the Simplest Assumptions for a Warranty Analysis?
   1. Known Population "Typical Value" mu
   2. Known Population Precision sigma
   3. Normal Distribution
   4. Randomness/Un-Autocorrelated
 
----------------------------------------------------------
 
11.2
What % of Units will Fail Warranty?
      LET MU = 100000
      LET SIGMA = 10000
      LET YCUTOFF = 70000
      LET ZCUTOFF = (YCUTOFF-MU)/SIGMA
      LET PROPFAIL = 100*NORCDF(ZCUTOFF)
 
----------------------------------------------------------
 
11.3
How Choose a Warranty Value so at most p% of Units Fail?
      LET MU = 100000
      LET SIGMA = 10000
      LET PROPFAIL = 5
      LET P = PROPFAIL/100
      LET ZCUTOFF=NORPPF(P)
      LET XCUTOFF = MU+ZCUTOFF*SIGMA
 
----------------------------------------------------------
 
 
 
 
 
 
 
 
 
 
 
 
 
 
 
 
 
 
 
 
 
 
 
 
 
 
 
 
 
 
 
 
 
 
 
 
 
 
 
 
 
 
 
 
 
 
 
 
 
 
 
 
 
 
 
 
 
 
 
 
 
----------  *PROCESS CAPABILITY ANALYSIS*  ----------------------
 
12
Proc. Capability Anal.--How "Good"/Capable is the Process?
   1. What are the Simplest Assumptions for a Proc. Capability Anal.?
   2. How Compute Capability Statistics?
   3. Is "Typical" Process Yield the Same for All Levels of a Factor?
   4. How Generate a Histogram with Tolerance Limits?
   5. How Generate a Histogram with Superimposed Normal Density?
 
----------------------------------------------------------
 
12.1
What are the Simplest Assumptions for a Process Capability Anal.?
   1. Known Population "Typical Value" mu
   2. Known Population Precision sigma
   3. Normal Distribution
   4. Randomness/Un-Autocorrelated
 
----------------------------------------------------------
 
12.2
How Compute Capability Statistics?
      LET LOWER = 10
      LET UPPER = 14
      LET CP = CP Y
      LET CPK = CPK Y
      LET PD = PERCENT DEFECTIVE Y
      LET
      LET EL = EXPECTED LOSS Y
 
----------------------------------------------------------
 
12.3
Are Capability Statistics the Same for All Levels of a Factor?
         LET LOWER =
         LET UPPER =
         CP PLOT Y X
         CPK PLOT Y X
         PERCENT DEFECTIVE PLOT Y X
         LET COST =
         EXPECTED LOSS PLOT Y X
 
----------------------------------------------------------
 
12.4
How Generate a Histogram with Tolerance Limits?
 
----------------------------------------------------------
 
12.5
How Generate a Histogram with Superimposed Normal Density?
 
----------------------------------------------------------
 
 
 
 
 
 
 
 
 
 
 
 
 
 
 
 
 
 
 
 
 
 
 
 
 
 
 
 
 
 
 
 
 
 
 
 
 
 
 
 
 
 
 
 
 
 
----------  *QUALITY ANALYSIS*  ----------------------
 
13
Quality Analysis--How Can the Process be Improved?
   1. What are the 7 "Old" Tools for Quality Improvement?
   2. What are the 7 "New" Tools for Quality Improvement?
   3. What is Quality Function Deployment (QFD)?
 
----------------------------------------------------------
 
13.1
What are the 7 "Old" Tools for Quality Improvement?
   1. Check Sheet
   2. Pareto Diagram
   3. Ishikawa Diagram
   4. Histogram
   5. Stratified Histogram
   6. Scatter Plot
   7. Control Charts
 
----------------------------------------------------------
 
13.1.1
Check Sheet
 
----------------------------------------------------------
 
13.1.2
Pareto Diagram
      PARETO PLOT Y TAG
 
----------------------------------------------------------
 
13.1.3
Ishikawa Diagram
      CALL ISHIKAWA.DP
 
----------------------------------------------------------
 
13.1.4
Histogram
      HISTOGRAM Y
 
----------------------------------------------------------
 
13.1.5
Stratified Histogram
      LET Y1 = Y
      LET Y2 = Y
      RETAIN Y1 SUBSET X 1
      RETAIN Y2 SUBSET X 2
      BIHISTOGRAM Y1 Y2
 
----------------------------------------------------------
 
13.1.6
Scatter Plot
      PLOT Y X
 
----------------------------------------------------------
 
13.1.7
Control Charts
      XBAR CONTROL CHART Y X
      R CONTROL CHART Y X
      STANDARD DEVIATION CONTROL CHART Y X
      C CONTROL CHART Y X
      P CONTROL CHART Y X
      NP CONTROL CHART Y X
      U CONTROL CHART Y X
 
----------------------------------------------------------
 
13.2
What are the 7 "New" Tools for Quality Improvement?
   1. Relations Diagram
   2. Affinity Diagram
   3. Systematic Diagram
   4. Matrix Diagram
   5. Matrix Data-Analysis
   6. Process Decision Program Chart (PDPC)
   7. Arrow Diagram Method
 
      Dataplot has no formal commands for the "7 New Tools".
      A good reference for details is
            Mizuno, Shigeru (1979).  Management for Quality
            Improvement:  The Seven New QC Tools.  Productivity
            Press, Cambridge, Mass.
 
----------------------------------------------------------
 
13.3
What is Quality Function Deployment (QFD)?
 
      QFD is an approach/technique for collecting
      and organizing engineering and customer information.
      It "forces" the engineer to address the "voice of
      the customer" with the net effect that all customer
      concerns are addressed and prioritized.
      DATAPLOT has no formal QFD command.
 
      A good (short) reference is
            Sullivan, Lawrence P. (1988).  Quality Function
            Deployment.  Qaulity Progress, June 1986,
            pages 39-50.
 
----------------------------------------------------------
 
 
 
 
 
 
 
 
 
 
 
 
 
 
 
 
 
 
 
 
 
 
 
 
 
 
 
 
 
 
 
 
 
 
 
 
 
 
 
 
 
 
 
 
 
 
 
 
 
 
 
 
 
 
 
 
 
 
 
 
 
 
 
 
 
 
 
 
 
 
 
 
 
 
 
 
 
 
 
 
 
 
 
 
 
 
 
 
 
 
 
 
 
----------  *TIME SERIES ANALYSIS*  ----------------------
 
14
Time Series Analysis  --How Analyze Equi-spaced Data?
   1. How Examine a Univariate Time Series?
   2. How Examine a Bivariate  Time Series?
 
----------------------------------------------------------
 
14.1
How Examine a Univariate Time Series?
   1. How Determine Time-Domain Relatedness?
   2. How Estimate Parameters in Autoregressive Model?
   3. How Estimate Parameters in ARIMA Model?
   4. How Determine Frequency-Domain Relatedness?
   5. How Determine Low-Frequency-Domain Model?
   6. Constant Amplitude & Phase in Single-Cycle Sine Model?
   7. Is Autocorrelation the Same for All Levels of a Factor
   8. How Smooth the Data?
 
----------------------------------------------------------
 
14.1.1
How Determine Time-Domain Relatedness?
      LAG PLOT Y
      AUTOCORRELATION PLOT Y
 
----------------------------------------------------------
 
14.1.2
How Estimate Parameters in Autoregressive Model?
 
----------------------------------------------------------
 
14.1.3
How Estimate Parameters in ARIMA Models?
 
----------------------------------------------------------
 
14.1.4
How Determine Frequency-Domain Relatedness?
      SPECTRAL PLOT Y
      PERIODOGRAM Y
 
----------------------------------------------------------
 
14.1.5
How Determine Low-Frequency-Domain Model?
      ALLAN VARIANCE PLOT Y
      ALLAN STANDARD DEVIATION PLOT Y
 
----------------------------------------------------------
 
14.1.6
Constant Amplitude & Phase in Single-Cycle Sine Model?
      SPECTRUM Y
      DEMODULATION FREQUENCY <some value from 0 to .5 from spectrum>
      COMPLEX DEMODULATION AMPLITUDE PLOT Y
      COMPLEX DEMODULATION PHASE PLOT Y
 
----------------------------------------------------------
 
14.1.7
Is Autocorrelation the Same for All Levels of a Factor
      AUTOCORRELATION STATISTIC PLOT Y
 
----------------------------------------------------------
 
14.1.8
How Smooth/Filter the Data?
 
   1. Time Domain Smoothing (Low-Pass Filtering) (Least Squares)
   2. Time Domain Smoothing (Low-Pass Filtering) (Robust)
   3. Frequency Domain Smoothing (Low-Pass Filtering)
 
   4. Time Domain High-Pass Filtering (Least Squares)
   5. Time Domain High-Pass Filtering (Robust)
   6. Frequency Domain High-Pass Filtering
 
   7. Time Domain Band-Pass Filtering (Least Squares)
   8. Time Domain Band-Pass Filtering (Robust)
   9. Frequency Domain Band-Pass Filtering
 
----------------------------------------------------------
 
14.1.8.1
Time Domain Smoothing (Low-Pass Filtering) (Least Squares)
 
----------------------------------------------------------
 
14.1.8.2
Time Domain Smoothing (Low-Pass Filtering) (Robust)
 
----------------------------------------------------------
 
14.1.8.3
Frequency Domain Smoothing (Low-Pass Filtering)
 
----------------------------------------------------------
 
14.1.8.4
Time Domain High-Pass Filtering (Least Squares)
 
----------------------------------------------------------
 
14.1.8.5
Time Domain High-Pass Filtering (Robust)
 
----------------------------------------------------------
 
14.1.8.6
Frequency Domain High-Pass Filtering
 
----------------------------------------------------------
 
14.1.8.7
Time Domain Band-Pass Filtering (Least Squares)
 
----------------------------------------------------------
 
14.1.8.8
Time Domain Band-Pass Filtering (Robust)
 
----------------------------------------------------------
 
14.1.8.9
Frequency Domain Band-Pass Filtering
 
----------------------------------------------------------
 
14.2
How Examine a Bivariate Time Series?
   1. How Determine Time-Domain Relatedness?
   2. How Estimate Parameters in Autoregressive Model?
   3. How Determine Frequency-Domain Relatedness?
 
----------------------------------------------------------
 
14.2.1
How Determine Time-Domain Relatedness?
      LAG ... PLOT Y1 Y2
         LAG PLOT Y1 Y2
         LAG 1 PLOT Y1 Y2
         LAG 2 PLOT Y1 Y2
         LAG 3 PLOT Y1 Y2
         LAG <any positive integer> PLOT Y1 Y2
      CROSS-CORRELATION PLOT Y1 Y2
 
----------------------------------------------------------
 
14.2.2
How Estimate Parameters in Bi-regressive Model?
 
----------------------------------------------------------
 
14.2.3
How Determine Frequency-Domain Relatedness?
      CROSS-SPECTRAL PLOT Y1 Y2
      CO-SPECTRAL PLOT Y1 Y2
      QUADRATURE SPECTRAL PLOT Y1 Y2
      COHERENCY SPECTRAM PLOT Y1 Y2
      AMPLITUDE SPECTRAL PLOT Y1 Y2
      GAIN SPECTRAL PLOT Y1 Y2
      ARGAND SPECTRAL PLOT Y1 Y2
      PHASE SPECTRAL PLOT Y1 Y2
 
-----------------------------------------------------
 
 
 
 
 
 
 
 
 
 
 
 
 
 
 
 
 
 
 
 
 
 
 
 
 
 
 
 
 
 
 
 
 
----------  *REGRESSION ANALYSIS*  ----------------------
 
15
Regression Analysis--How Determine the Model in Y = f(X1,X2,...)?
   1. Regression Assumptions
   2. General Regression Procedure
   3. Fit 1-Variable Model y = f(x)
   4. Fit 2-Variable Model y = f(x1,x2)
   5. Fit 3/4/5/...-Variable Model y = f(x1,x2,x3,...) (Multivar. Reg.)
 
----------------------------------------------------------
 
15.1
Regression Assumptions
   1. Correct, Complete, Fixed Model
   2. Fixed Precision/Variation for Errors
   3. Fixed Distribution for Errors
   4. Random, Un-Autocorrelated Errors
 
----------------------------------------------------------
 
15.2
General Regression Procedure
   1. Model Selection (Plot the Data--PLOT)
   2. Parameter Estimation (Fit the Data--FIT)
   3. Model Validation (Residual Analysis--PLOT)
 
----------------------------------------------------------
 
15.3
Fit 1-Variable Model y = f(x)
   1. How Select the Proper Model?
   2. How Fit a Linear, Quadratic, or Polynomial model?
   3. How Fit a General Model (Linear, Polynomial, Non-Linear, etc.)?
   4. How Fit an Exponential Model?
   5. How Fit a Sinusoidal Model?
   6. How Do Residual Analysis?
   7. How Transform to Linear Model?
   8. How do Robust Regression?
   9. How do Weighted Regression?
 
-----------------------------------------------------
 
15.3.1
How Select the Proper Model?
      PLOT Y X
      LET C1 = CORRELATION Y X
      LET C2 = RANK CORRELATION Y X
 
-----------------------------------------------------
 
15.3.2
How Fit a Linear, Quadratic, or Polynomial Model?
      ... FIT Y X
         LINEAR FIT Y X
         QUADRATIC FIT Y X
         CUBIC FIT Y X
         QUARTIC FIT Y X
         QUINTIC FIT Y X
         SEXTIC FIT Y X
         SEPTIC FIT Y X
         OCTIC FIT Y X
         NONIC FIT Y X
         DEXIC FIT Y X
 
-----------------------------------------------------
 
15.3.3
How Fit a General Model (Linear, Polynomial, Non-Linear, etc.)?
      Fit Y = <any Fortran-like expression>
 
-----------------------------------------------------
 
15.3.4
How Fit a Rational Function Model?
      EXACT .../... Rational FIT YSUB XSUB Y X
         EXACT 1/1 RATIONAL FIT YSUB XSUB Y X
         EXACT 1/2 RATIONAL FIT YSUB XSUB Y X
         EXACT 2/1 RATIONAL FIT YSUB XSUB Y X
         EXACT 2/2 RATIONAL FIT YSUB XSUB Y X
         EXACT <etc.> RATIONAL FIT YSUB XSBU Y X
      FIT Y = ...
 
-----------------------------------------------------
 
15.3.5
How Fit an Exponential Model?
      FIT Y = A+B*EXP(C*X)
 
-----------------------------------------------------
 
15.3.6
How Fit a Sinusoidal Model?
      PLOT Y X
      LAG PLOT Y
      SPECTRUM Y
      LET MU = MIDRANGE Y
      LET R = RANGE Y
      LET AMP = R/2
      LET THETA = <value between 0 and .5 from SPECTRAL PLOT>
      FIT Y = MU + AMP*SIN(2*3.14159*THETA*X+PHASE)
 
-----------------------------------------------------
 
15.3.7
How Do Residual Analysis?
      FIT Y = ...
      CHARACTERS X BLANK
      LINES BLANK SOLID
      PLOT Y PRED VS X
      PLOT RES X
      4-PLOT RES
 
-----------------------------------------------------
 
15.3.8
How Transform to Linear Model?
      BOX-COX LINEARITY PLOT Y X
 
-----------------------------------------------------
 
15.3.9
How do Robust regression?
 
-----------------------------------------------------
 
15.3.10
How to Weighted Regression?
 
-----------------------------------------------------
 
15.3.11
How do Non-Least Squares Regression?
 
----------------------------------------------------------
 
15.4
Fit 2-Variable Model y = f(x1,x2)
   1. How Select the Proper Model?
   2. How Fit a 2-Variable Multilinear Model?
   3. How Fit a General Model (Multi-Linear, Non-Linear, etc.)?
   4. How Fit an Exponential Model in 2 Variables?
   5. How Do Residual Analysis?
   6. How do Robust Regression?
   7. How do Weighted Regression?
 
-----------------------------------------------------
 
15.4.1
How Select the Proper Model?
      PLOT Y X1
      PLOT Y X2
      LET C1 = CORRELATION Y X1
      LET C2 = CORRELATION Y X2
      LET RC1 = RANK CORRELATION Y X1
      LET RC2 = RANK CORRELATION Y X2
 
-----------------------------------------------------
 
15.4.2
How Fit a 2-Variable Multilinear Model?
      FIT Y X1 X2
      FIT Y = B0 + B1*X1 + B2*X2
      FIT Y = B0 + B1*X1 + B2*X2 + B12*X1*X2
      FIT Y = B0 + B1*X1 + B2*X2 + B12*X1*X2 + B11*X1**2
 
-----------------------------------------------------
 
15.4.3
How Fit a General Model (Multi-Linear, Non-Linear, etc.)?
      Fit Y = <any Fortran-like expression>
 
-----------------------------------------------------
 
15.4.4
How Fit an Exponential Model in 2 Variables?
      FIT Y = B0 + B1*EXP(C1*X1) + B2*EXP(C2*X2)
 
-----------------------------------------------------
 
15.4.5
How Do Residual Analysis?
      FIT Y = ...
      CHARACTERS X BLANK
      LINES BLANK SOLID
      PLOT Y PRED VS X1
      PLOT Y PRED VS X2
      PLOT RES X1
      PLOT RES X2
      4-PLOT RES
 
-----------------------------------------------------
 
15.4.6
How do Robust Regression?
 
-----------------------------------------------------
 
15.4.7
How to Weighted Regression?
 
----------------------------------------------------------
 
15.5
Fit 3/4/5/...-Variable Model y = f(x1,x2,x3,...) (Multivariate Reg.)
   1. How Select the Proper Model?
   2. How Fit a k-Variable Multilinear Model?
   3. How Do All Possible Subsets Regression?
   4. How Fit a General Model (Multi-Linear, Non-Linear, etc.)?
   5. How Fit an Exponential Model in k Variables?
   6. How Do Residual Analysis?
   7. How do Robust Regression?
   8. How do Weighted Regression?
 
-----------------------------------------------------
 
15.5.1
How Select the Proper Model?
      PLOT Y X1
      PLOT Y X2
      LET C1 = CORRELATION Y X1
      LET C2 = CORRELATION Y X2
      LET RC1 = RANK CORRELATION Y X1
      LET RC2 = RANK CORRELATION Y X2
 
   Correlation
   Partial Correlation
   Multiplotting
   All Possible Subsets Regression
-----------------------------------------------------
 
15.5.2
How Fit a k-Variable Multilinear Model?
      FIT Y X1 X2 X3 X4 X5
 
-----------------------------------------------------
 
16.5.3
How Do All Possible Subsets Regression?
 
-----------------------------------------------------
 
15.5.4
How Fit a General Model (Multi-Linear, Non-Linear, etc.)?
      Fit Y = <any Fortran-like expression>
 
-----------------------------------------------------
 
15.5.5
How Fit an Exponential Model in k Variables?
 
-----------------------------------------------------
 
15.5.6
How Do Residual Analysis?
      FIT Y = ...
      CHARACTERS X BLANK
      LINES BLANK SOLID
.
      MULTIPLOT 2 3
      PLOT Y PRED VS X1
      PLOT Y PRED VS X2
      PLOT Y PRED VS X3
      PLOT Y PRED VS X4
      PLOT Y PRED VS X5
.
      MULTIPLOT 2 3
      PLOT RES X1
      PLOT RES X2
      PLOT RES X3
      PLOT RES X4
      PLOT RES X5
.
      MULTIPLOT 4 5
      LOOP FOR J = 1 1 K
      PLOT Y PRED VS X^J
      END OF LOOP
 
      MULTIPLOT 4 5
      LOOP FOR J = 1 1 K
      PLOT RES X^J
      END OF LOOP
.
      4-PLOT RES
 
-----------------------------------------------------
 
15.5.7
How do Robust Regression?
 
-----------------------------------------------------
 
15.5.8
How to Weighted Regression?
 
 
 
 
 
 
----------  *MULTIFACTOR ANALYSIS*  ----------------------
 
16
Multifactor Analysis--What Factors Affect Process Yield?
(What are the Important Process Variables?)
   Y Affected by 1 Factor --X1?
   Y Affected by 2 Factors--X1, X2?
   Y Affected by 3 Factors--X1, X2, X3?
   Y Affected by 4 Factors--X1, X2, X3, X4?
   Y Affected by 5 Factors--X1, X2, X3, X4, X5?
   Y Affected by 6 Factors--X1, X2, X3, X4, X5, X6?
 
How Analyze Output from a Designed Experiment?
   Comparative Designs
   Factorial Designs
   Regression Designs
   Response Surface Designs
   Simplex Designs
 
 
 
 
 
 
 
 
 
 
 
 
 
 
 
 
 
 
 
 
 
 
 
 
 
 
 
 
 
 
 
 
 
 
 
 
 
 
 
 
 
 
 
 
 
 
 
 
 
 
 
 
 
 
 
 
 
 
 
 
 
 
 
 
 
 
 
 
 
 
 
 
 
 
 
 
 
 
 
 
 
 
----------  *MULTIVARIATE ANALYSIS*  ----------------------
 
17
Multivariate Analysis--How Examine Multiple Responses?
   1. How Examine 2 Responses--Y1 and Y2?
   2. How Examine 3/4/5/... Responses--Y1, Y2, Y3,...?
   3. How Do Cross-Correlation?
   4. How Do Discrimination Analysis?
   5. How Do Cluster Analysis?
   6. How Do Cross-Tabulation?
   7. How Do Canonical Analysis?
   8. How Do Principal Components Analysis?
 
-----------------------------------------------------
 
17.1
How Examine 2 Responses--Y1 and Y2?
      PLOT Y1 Y2
      BIHISTOGRAM Y1 Y2
      QUANTILE QUANTILE PLOT Y1 Y2
      PLOT Y1 Y2 TAG
      YOUDEN PLOT Y1 Y2 TAG
 
-----------------------------------------------------
 
17.2
How Examine 3/4/5/... Responses--Y1, Y2, Y3,...?
   PLOT Y1 Y2 Y3 Y4 ... VERSUS X
   PLOT Y X TAG
   MULTIPLOT
   PROFILE PLOT Y1 Y2 Y3 ETC.
   STAR PLOT Y1 Y2 Y3 ETC.
 
-----------------------------------------------------
 
17.3
How Do Cross-Correlation?
 
-----------------------------------------------------
 
17.4
How Do Discrimination Analysis?
 
-----------------------------------------------------
 
17.5
How Do Cluster Analysis?
 
-----------------------------------------------------
 
17.6
How Do Cross-Tabulation?
 
-----------------------------------------------------
 
17.7
How Do Cannonical Analysis?
 
-----------------------------------------------------
 
17.8
How Do Principal Components Analysis?
 
-----------------------------------------------------
 
 
 
 
 
 
 
 
 
 
 
 
 
 
 
 
 
 
 
 
 
 
 
 
 
 
 
 
 
 
 
 
 
 
 
 
----------  *INTERLAB ANALYSIS*  ----------------------
 
18
Interlab Analysis--Are All Labs Equally Good?
      LINES BLANK ALL
      CHARACTERS 1 2 3 4 5 6 7 8 9 A B C D E F G H I JK
      YOUDEN PLOT Y1 Y2 LAB
      PLOT Y1 Y2 LAB
      PLOT Y1 Y2 VERSUS LAB
 
----------------------------------------------------------
 
 
 
 
 
 
 
 
 
 
 
 
 
 
 
 
 
 
 
 
 
 
 
 
 
 
 
 
 
 
 
 
 
 
 
 
 
 
 
 
 
 
 
 
 
 
 
 
 
 
 
 
 
 
 
 
 
 
 
 
 
 
 
 
 
 
 
 
 
 
 
 
 
 
 
 
 
 
 
 
 
 
 
 
 
 
 
 
 
----------------------------------------------------------
THE MATERIAL BELOW WAS NOT USED
 
If Assumptions Fail--Transformational Analysis
   BOX-COX HOMOSCEDASTICITY PLOT Y X
   BOX-COX NORMALITY PLOT Y
   BOX-COX LINEARITY PLOT Y X
 
Robust & Distribution-Free Analysis
   Robust Analysis
   Distribution-Free Analysis
 
Statistical graphics
Summary graphics
 
Graphical data analysis/EDA
Graphical residual analysis
Set multiplot
Capture next plot only
Capture all succeeding plots
Set graphics device
Examples
Show graphics gallery
Help
~ANOVA~
~Distributional Analysis~
~EDA~
~Estimation~
~Fitting~
~Gallery~
~Graphical~
~Interlab~
~Location Est./Test.~
~Multivatiate~
~Non-parametric~
~Reliability~
~Random Numbers~
~Residual Analysis~
~Robust Analysis~
~Smoothing~
~SPC~
~Subset Analyses~
~Summaries~
~Testing~
~Times Series~
~Transformations~
~Underlying Assumptions~
~Variation Est./Test.~
~Your First Analysis~
~Your First Fit~
~Your First Histogram~
~Your First Non-Linear Fit~
~Your First Time Series Analysis~
~Your First Weibull Plot~
Your First ...
Fit
Stat Computation
Variable Transformation
5.----------Probability----------
~Gallery~
Your First ...
Probability Calculation
6.----------Mathematics----------
~Arithmetic Operations~
~Built-in Functions~
~Complex Arithmetic~
~Complex Roots~
~Concatonating Functions~
~Convolution~
~Creating Functions~
~Defining Functions~
~Differentiation~
~Diff. Eq. Solver~
~Fitting Functions~
~FFT~
~Fractal Plots~
~Functions~
~Gallery~
~Integration~
~Logical Set Operations~
~Matrix Operations~
~Operations~
~Plotting Functions~
~Poincare Plots~
~Polynomial Operations~
~Roots~
~Set Operations~
~Simplex LP Solver~
~Your First FFT~
Your First ...
Root Calculation
Matrix Operation
Fast Fourier Transform
///////
FRACTAL PLOT
... FREQUENCY PLOT
PLOT
 
 
Summary statistics
Distributional statistics
Graphical data analysis/EDA
Generate data internally
Define variables
Define parameters
Define functions/strings
Evaluate a function
Compute an expression
Create a new variable
Quantitative Statistics
Error Bar Analysis
ERROR BAR PLOT Y X
 
What Process Inspection Plan to Use?
Acceptance Sampling
   Introduction
   OC (Operating Characteristic) Curves
   Sampling Plans
      By Quality Index
         AQL (Acceptance Quality Level) Plans
            MIL-STD-105D
            MIL-STD-414
         LTPD (Lot Tolerance Percent Defective) Plans
         AOQL (Average Outgoing Quality Limit) Plans
      By Inspection Type
         Variable
         Attribute
      By Number of Stages
         Single-Stage
         Double-Stage
         Multi-Stage
   Minimizing Inspection Costs
   Minimizing Repair Costs
 
Business Graphics
   PIE CHART Y
   HISTOGRAM Y
   PARETO PLOT Y
   PLOT
 
Analysis of 2-way Tables
Contingency Tables
Analysis of Covariance
/////
Assumptions           --What are Underlying Assumptions of a Proc.?
Process Summarization --How Summarize the Output from a Process?
Location Analysis     --What is the "Typical Value" of a Process?
Variation Analysis    --What is the Precision of a Process?
Randomness Analysis   --How Random/Un-autocorrelated is the Process?
Stability Analysis    --Is Process Same for All Levels of a Factor?
Uncertainty Analysis  --How Attach Error Bars to an Estimator?
Stat. Proc. Control   --Is the Process Statistically "In Control"?
Reliability Analysis  --What is Process/Product Lifetime?
Warranty Analysis     --How Determine a Warrenty Value for a Product?
Proc. Capability Anal.--How "Good"/Capable is the Process?
Quality Analysis      --How Can the Process be Improved?
Time Series Analysis  --How Analyze Equi-spaced Data?
Regression Analysis   --How Determine the Model in Y = f(X1,X2,...)?
Multifactor Analysis  --What Factors Affect Process Yield?
Multivariate Analysis --How Examine Multiple Responses?
Interlab Analysis     --Are All Labs Equally Good?
500.      0           0. Initial Menu
600.      1               1. Univariate Analysis
1028.     1.1                   1. General Discussion
1037.     1.2                   2. Computing Summary/Descriptive Statistics
1051.     1.2.1                       1. Location Estimation
1078.     1.2.2                       2. Variation (Scale) Estimation
1106.     1.2.3                       3. Skewness Estimation
1133.     1.2.4                       4. Tail Length Estimation
1156.     1.2.5                       5. Autocorrelation estimation
1182.     1.3                   3. Determining General Distributional Characteri
1201.     1.3.1                       1. Frequency Tabulation
490.      1.3.2                       2. Histograms and Cumulative Histograms
490.      1.3.3                       3. Stem and Leaf Diagrams
490.      1.3.4                       4. Frequency Plots and Cumulative Frequenc
490.      1.3.5                       5. Percent Point Plots
490.      1.3.6                       6. Pie Charts
490.      1.4                   4. Selecting a "Good-Fitting" Distribution
490.      1.4.1                       1. Probability Plots
490.      1.4.2                       2. PPCC Plots
490.      1.4.3                       3. Maximum Likelihood Estimation
490.      1.5                   5. Estimating the Parameters of the Distribution
490.      1.5.1                       1. Maximum Likelihood Estimation
490.      1.5.2                       2. Robust Estimates
490.      1.6                   6. Assessing the Goodness of Fit of a Distributi
490.      1.6.1                       1. Superimposing Probability Density Funct
490.      1.6.2                       2. Superimposing Root Density Function and
490.      1.6.3                       3. Probability Plot
490.      1.6.4                       4. Chi-Squared Test
490.      1.6.5                       5. Kolmogorov-Smirnoff Test
490.      1.7                   7. Testing Underlying Assumptions
490.      1.7.1                       1. 4-Plot Analysis
490.      1.8                   8. Testing for Randomness
490.      1.8.1                       1. Lag Plot
490.      1.8.2                       2. Runs Analysis
490.      1.8.3                       3. Distribution-Free Tests
490.      1.8.4                       4. Autocorrelation Plot
490.      1.8.5                       5. Spectral Plot
490.      1.9                   9. Testing for Fixed Location (No Shifts in Loca
490.      1.9.1                       1. t Test
490.      1.9.2                       2. Distribution-Free Tests
490.      1.10                 10. Testing for Fixed Variation (Homoscedasticity
490.      1.10.1                      1. Homoscedasticity Plot
490.      1.11                 11. Transforming to Homoscedasticity
490.      1.11.1                      1. Box-Cox Homoscedasticity Plot
490.      1.11.2                      2. Chi-squared Tests
490.      1.12                 12. Testing for Fixed Distribution
490.      1.12.1                      1. Bihistogram
490.      1.12.2                      2. 4-Plot Analysis
490.      1.12.3                      3. Distribution-Free Tests
490.      1.12.4                      4. Homoscedasticity Plot
490.      1.13                 13. Testing for Symmetry
490.      1.13.1                      1. Symmetry Plot
490.      1.14                 14. Transforming to Symmetry
490.      1.14.1                      1. Box-Cox Symmetry Plot
490.      1.15                 15. Testing for Normality
490.      1.15.1                      1. Normal Probability Plot
490.      1.15.2                      2. Tukey PPCC Plot
490.      1.15.3                      3. t PPCC Plot
490.      1.16                 16. Transforming to Normality
490.      1.16.1                      1. Box-Cox Normality Plot
490.      1.17                 17. Testing for Normal Outliers
490.      1.18                 18. Computing Confidence Limits for Distributiona
490.      1.19                 19. Hypothesis Testing on Distributional Paramete
1500.     2               2. Time Series Analysis (1 Variable)
490.      2.1                   1. General discussion
490.      2.2                   2. Checking for Time-Domain Structure
490.      2.2.1                       1. Run Sequence Plot
490.      2.2.2                       2. Lag Plot
490.      2.2.3                       3. Autocorrelation Plot
490.      2.2.4                       4. Partial-Autocorrelation Plot
490.      2.2.5                       5. Complex Demodulation Plots
490.      2.3                   3. Checking for Frequency-Domain Structure
490.      2.3.1                       1. Spectral Plot
490.      2.3.2                       2. Periodogram
490.      2.4                   4. Checking for Time and Frequency Domain Struct
490.      2.4.1                       1. 4-Plot Analysis
490.      2.5                   5. Testing White Noise (Randomness)
490.      2.5.1                       1. Lag Plot
490.      2.5.2                       2. Runs Analysis
490.      2.5.3                       3. Distribution-Free Tests
490.      2.5.4                       4. Autocorrelation Plot
490.      2.5.5                       5. Spectral Plot
490.      2.5.6                       6. 4-Plot Analysis
490.      2.6                   6. Checking for Trends
490.      2.6.1                       1. Run Sequence Plot
490.      2.6.2                       2. Correlation With Time
490.      2.6.3                       3. Linear Fit Over Time
490.      2.7                   7. Fitting Box-Jenkins Models
490.      2.7.1                       1. Lag Plot
490.      2.7.2                       2. Autocorrelation Plot
490.      2.7.3                       3. Partial Autocorrelation Plot
490.      2.8                   8. Smoothing
490.      2.8.1                       1. Moving Average Smoothing
490.      2.8.2                       2. Least Squares Smoothing
490.      2.8.3                       3. Median Smoothing
490.      2.8.4                       4. Robust Smoothing
490.      2.8.5                       5. Exponential Smoothing
490.      2.8.6                       6. Assessing the Goodness of the Smoothing
490.      2.8.6.1                           1. Residual Standard Deviation
490.      2.8.6.2                           2. Superimposing Raw Data and Fitted
490.      2.8.6.3                           3. Scatter Plots of Residuals
490.      2.8.6.4                           4. Normal Probability Plot of Residu
490.      2.8.6.5                           5. 5-Plot of Residuals
490.      2.9                   9. Filtering
490.      2.9.1                       1. Low-Pass Filters
490.      2.9.2                       2. High-Pass Filters
490.      2.9.3                       3. Assessing the Goodness of the Filtering
1700.     3               3. Time Series Analysis--2 Variables
490.      3.1                   1. General Discussion
490.      3.2                   2. Checking for Time-Domain Structure
490.      3.2.1                       1. Scatter Plot
490.      3.2.2                       2. Multi-Trace Plots
490.      3.2.3                       3. Cross-Spectral Plot
490.      3.2.4                       4. Bihistogram
490.      3.3                   3. Checking for Frequency-Domain Structure
490.      3.3.1                       1. Cross-Spectrum
490.      3.3.2                       2. Coherency Spectrum
490.      3.3.3                       3. Quadrature Spectrum
490.      3.3.4                       4. Co-Spectrum
490.      3.3.5                       5. Gain Spectrum
490.      3.3.6                       6. Argand Spectrum
490.      3.4                   4. Checking for Time and Frequency Domain Struct
490.      3.4.1                       1. 4-Plot Analysis
1800.     4               4. Correlation Analysis
490.      4.1                   1. General Discussion
490.      4.2                   2. Multi-Scatter Plots
490.      4.3                   3. Multi-ANOP Plots
490.      4.4                   4. Multi-Box Plots
490.      4.5                   5. Cross-Correlation Tabulation
490.      4.6                   6. Transforming Variables
490.      4.7                   7. Distribution-free Tests
1900.     5               5. Fitting (1 Independent Variable)
1917.     5.1                   1. General discussion
1954.     5.2                   2. Selecting a Model
1979.     5.2.1                       1. Plotting the Data
2009.     5.2.2                       2. Generate Reference Data Plots
2562.     5.2.2.1                           1. Shape 1--Quadratic
2592.     5.2.2.2                           2. Shape 2--Monotonic up
2622.     5.2.2.3                           3. Shape 3--Monotonic up
2652.     5.2.2.4                           4. Shape 4--S-Shaped
2682.     5.2.2.5                           5. Shape 5--Quadratic
2712.     5.2.2.6                           6. Shape 6--Linear
2742.     5.2.2.7                           7. Shape 7--Square Root
2772.     5.2.2.8                           8. Shape 8--Asymptote
2802.     5.2.2.9                           9. Shape 9--Cubic
2832.     5.2.2.10                         10. Shape 10--Monotonic
2862.     5.2.2.11                         11. Shape 11--Logarithm
2892.     5.2.2.12                         12. Shape 12--Asymptote
2922.     5.2.2.13                         13. Shape 13--Skewed
2952.     5.2.2.14                         14. Shape 14--Skewed
2982.     5.2.2.15                         15. Shape 15--Quadratic
3012.     5.2.2.16                         16. Shape 16--Skewed
3042.     5.2.2.17                         17. Shape 17--Bell-Shaped
3072.     5.2.2.18                         18. Shape 18--Hyberbolic
3102.     5.2.2.19                         19. Shape 19--Neg. Exponential
3132.     5.2.2.20                         20. Shape 20--Z-Shaped
3162.     5.3                   3. Fitting a Model
3174.     5.3.1                       1. Fitting Linear Models
490.      5.3.2                       2. Fitting Polynomial Models
490.      5.3.3                       3. Fitting Non-Linear Models
490.      5.3.4                       4. Fitting Rational Functions
3218.     5.3.5                       5. Fitting Splines
3254.     5.4                   4. Assessing the Goodness of Fit of the Model
490.      5.4.1                       1. Residual Standard Deviation
490.      5.4.2                       2. Lack of Fit F Tests
3297.     5.4.3                       3. Superimposing Raw Data and Fitted Curve
3330.     5.4.4                       4. 4-Plot of Residuals
3366.     5.4.5                       5. Scatter Plots of Residuals
3399.     5.4.5.1                           1. Reference Scatter Plots of Residu
490.      5.4.5.1.1                               1  Residual Scatter Plot--Idea
490.      5.4.5.1.2                               2  Residual Scatter Plot--Wedg
490.      5.4.5.1.3                               3  Residual Scatter Plot--Wedg
490.      5.4.5.1.4                               4  Residual Scatter Plot--Wedg
490.      5.4.5.1.5                               5  Residual Scatter Plot--Line
490.      5.4.5.1.6                               6  Residual Scatter Plot--Quad
490.      5.4.5.1.7                               7  Residual Scatter Plot--Quad
3670.     5.4.5.1.8                               8  Residual Scatter Plot--Spli
490.      5.4.6                       6. Lag Plot of Residuals
490.      5.4.7                       7. Histograms of Residuals
490.      5.4.8                       8. Normal Probability Plot of Residuals
490.      5.5                   5. Improving the Model
490.      5.5.1                       1. Transforming to Simplify the Model
490.      5.5.2                       2. Transforming to Achieve Homogeneity
490.      5.5.3                       3. Transforming to Achieve Normality
490.      5.5.4                       4. Adding New Variables
490.      5.5.5                       5. Changing the Form of the Model
490.      5.6                   6. Fitting With Weights
490.      5.7                   7. Fitting With Constraints
490.      5.7                   8. Fitting With Other Criteria (e.g., L1 Fitting
4000.     6               6. Fitting (2 or More Independent Variable)
4018.     6.1                   1. General discussion
490.      6.2                   2. Selecting Variables for To Be Included in the
490.      6.2.1                       1. Multi-Run Sequence Plots
490.      6.2.2                       2. Multi-Histograms
490.      6.2.3                       3. Multi-Scatter Plots
490.      6.2.4                       4. Multi-ANOP Plots
490.      6.2.5                       5. Cross-Correlation Tabulation
490.      6.2.6                       6. Box Plots
490.      6.2.7                       7. Cp Plot
490.      6.3                   3. Selecting a Model
490.      6.3.1                       1. Plotting the Data
490.      6.3.2                       2. Making Use of Reference Curves
490.      6.4                   4. Fitting a Model
490.      6.4.1                       1. Fitting Multi-Linear Models
490.      6.4.2                       2. Fitting Non-Linear Models
490.      6.5                   5. Assessing the Goodness of Fit of the Model
490.      6.5.1                       1. Residual Standard Deviation
490.      6.5.2                       2. Lack of Fit F Tests
490.      6.5.3                       3. Superimposing Raw Data and Fitted Curve
490.      6.5.4                       4. Scatter Plots of Residuals
490.      6.5.5                       5. Normal Probability Plot of Residuals
490.      6.5.6                       6. 4-Plot of Residuals
490.      6.6                   6. Improving the Model
490.      6.6.1                       1. Transforming to Simplify the Model
490.      6.6.2                       2. Transforming to Achieve Homogeneity
490.      6.6.3                       3. Transforming to Achieve Normality
490.      6.6.4                       4. Adding New Variables
490.      6.6.5                       5. Changing the Form of the Model
490.      6.7                   7. Fitting With Weights
490.      6.8                   8. Fitting With Constraints
490.      6.8                   8. Fitting With Other Criteria (e.g., L1 Fitting
4200.     7               7. ANOVA Modeling
490.      7.1                   1. General discussion
490.      7.2                   2. Selecting Variables for To Be Included in the
490.      7.2.1                       1. Multi-Run Sequence Plots
490.      7.2.2                       2. Multi-Histograms
490.      7.2.3                       3. Multi-Scatter Plots
490.      7.2.4                       4. Multi-ANOP Plots
490.      7.3                   3. Examining 1-Factor Models
490.      7.3.1                       1. 1-Way ANOVA
490.      7.3.2                       2. 1-Way GANOVA
490.      7.3.3                       3. Scatter Plots
490.      7.3.4                       4. Box Plot
490.      7.3.5                       5. ANOP Line Plot
490.      7.3.6                       6. ANOP Character Plot
490.      7.3.7                       7. I Plot
490.      7.3.8                       8. Distribution-free Tests
490.      7.3.9                       9. Correlation
490.      7.3.10                     10. Categorical Data Analauysis
490.      7.3.11                     11. Cross-Corrleation
490.      7.3.12                     12. Discrete Contour Plot
490.      7.3.13                     13. Frequency Tabulation
490.      7.3.14                     14. Cross-Tabulation
490.      7.3.15                     15. Chi-squared
490.      7.4                   4. Examining 2-Factor Models
490.      7.4.1                       1. 2-Way ANOVA
490.      7.4.2                       2. 2-Way GANOVA
490.      7.4.3                       3. Median Polish
490.      7.4.4                       4. Multi-Trace Plots
490.      7.4.5                       5. 3-D Plot
490.      7.4.6                       6. Spike Plots
490.      7.5                   5. Examining 3-Factor Models
490.      7.5.1                       1. 3-Way ANOVA
490.      7.5.2                       2. 3-Way GANOVA
490.      7.5.3                       3. Multi-Cell Plots
490.      7.6                   6. Examining 4-factor Models
490.      7.6.1                       1. 4-Way ANOVA
490.      7.6.2                       2. Multi-Plot 2-Way GANOVA
490.      7.6.3                       3. Multi-Plot 3-Way GANOVA
490.      7.7                   7. Examining 5-Factor Models
490.      7.7.1                       1. 5-Way ANOVA
490.      7.7.2                       2. Multi-Plot 3-Way GANOVA
490.      7.8                   8. Examining 1-Factor Models With Only 2 Treatme
490.      7.8.1                       1. 1-Way ANOVA
490.      7.8.2                       2. t Test
490.      7.8.3                       3. Bihistogram
490.      7.9                   9. Examining 2**k Models
490.      7.9.1                       1. Square Plots, Cube Plots, etc.
490.      7.10                 10. Assessing the Goodness of Fit of the Model
490.      7.10.1                      1. Residual Standard Deviation
490.      7.10.2                      2. Lack of Fit F Tests
490.      7.10.3                      3. GANOVA, Parallelism, and Non-Additivity
490.      7.10.4                      4. Superimposing Raw Data and Fitted Curve
490.      7.10.5                      5. Scatter Plots of Residuals
490.      7.10.6                      6. Normal Probability Plot of Residuals
490.      7.10.7                      7. 4-Plot of Residuals
490.      7.11                 11. Improving the Model
490.      7.11.1                      1. Residual Standard Deviations For Sub-Mo
490.      7.11.2                      2. F Tests For Sub-Models
490.      7.11.3                      3. Transforming to Simplify the Model
490.      7.11.4                      4. Transforming to Achieve Additivity
490.      7.11.5                      5. Transforming to Achieve Homogeneity
490.      7.11.6                      6. Transforming to Achieve Normality
490.      7.11.7                      7. Omitting Variables From the Model
490.      7.11.7.1                          1. F Tests for Sub-Models
490.      7.11.8                      8. Selecting Additional Variables For the
490.      7.11.8.1                          1. Scatter Plots of Residuals on New
490.      7.11.8.2                          2. Box Plots of Residuals on New Var
490.      7.11.9                      9. Changing the Form of the Model
4600.     8               8. Multivariate Analysis
490.      8.1                   1. General Discussion
490.      8.2                   2. Cluster Analysis
490.      8.3                   3. Discriminant Analysis
490.      8.4                   4. Principal Component Analysis
490.      8.5                   5. Canonical Analysis
490.      8.6                   6. Testing Multivariate Normality--Q-Q Plot
490.      8.5
4700.     9               9. Probability Analysis
490.      9.1                   1. General Discussion
490.      9.2                   2. Generating Random Numbers/Simulation/Monte Ca
490.      9.3                   3. Computing Percent Points
490.      9.4                   4. Computing Probability Density Functions
490.      9.5                   5. Computing Cumulative Distribution Functions
490.      9.6                   6. Plotting Percent Points
490.      9.7                   7. Plotting Probability Density Functions
490.      9.8                   8. Plotting Cumulative Distribution Functions
490.      9.9                   9. Superimposing Probability Density Funtions on
4800.     10             10. Quality Control
490.      10.1                  1. General Discussion
490.      10.2                  2. Testing for Trends
490.      10.2.1                      1. Run Sequence Plot
490.      10.2.2                      2. Mean Control Chart
490.      10.3                  3. Testing for Shifts in Location
490.      10.3.1                      1. Run Sequence Plot
490.      10.3.2                      2. Mean Control Chart
490.      10.4                  4. Testing for Shifts in Variation
490.      10.4.1                      1. Range Control Chart
490.      10.4.2                      2. Standard Deviation Control Chart
490.      10.5                  5. Testing for Outliers
490.      10.6                  6. Interlaboratory Testing
490.      10.6.1                      1. Youden Plots
4900.     11             11. Distribution-Free Analysis
490.      11.1                  1. General Discussion
490.      11.2                  2. Testing for Randomness
490.      11.2.1                      1. Runs Analysis
490.      11.2.2                      2. Sign Test
490.      11.2.3                      3. Median Test
490.      11.3                  2. Testing for Fixed Location (No Shifts)
490.      11.3.1                      1. Sign Test
490.      11.4                  3. Testing for Fixed Variation (Homoscedasticity
490.      11.4.1                      1. Sign Test on First Differences
490.      11.5                  4. Testing for Goodness of Fit of a Distribution
490.      11.5.1                      1. Kolmogorov-Smirnoff Test
490.      11.6                  5. Testing for Correlation
490.      11.6.1                      1. Rank Correlation Coefficient
490.      11.7                  6. ANOVA Modeling
490.      11.7.1                      1. ANOVA On Ranks
490.      11.7.2                      2. Mann-Whitney Tests
5000.     12             12. Robust Analysis
490.      12.1                  1. General Discussion
490.      12.2                  2. Univariate Estimation
490.      12.2.1                      1. Robust Location Estimation
490.      12.2.2                      2. Robust Scale Estimation
490.      12.3                  2. Smoothing
490.      12.3.1                      1. Median Smoothing
490.      12.3.2                      2. Robust Smoothing
490.      12.4                  4. Fitting
490.      12.4.1                      1. L1 Fitting
490.      12.5                  5. ANOVA Modeling
490.      12.5.1                      1. ANOVA On Ranks
490.      12.5.2                      2. Median Polish
5100.     13             13. Exploratory Data Analysis
490.      13.1                  1. General Discussion
490.      13.2                  2. Run Sequence Plots
490.      13.3                  3. Lag Plots
490.      13.4                  4. Stem and Leaf Diagrams
490.      13.5                  5. Histograms
490.      13.6                  6. Normal Probability Plots
490.      13.7                  7. Scatter Plots
490.      13.8                  8. Box Plots
 
 
 
 
 
 
 
 
 
 
 
 
 
 
 
 
 
 
 
 
 
 
 
 
 
 
 
 
 
 
 
 
 
 
 
 
 
 
 
 
 
 
 
 
 
 
 
 
 
 
 
 
 
 
 
 
 
 
 
 
 
 
 
 
 
 
 
 
 
 
 
 
 
 
 
 
 
 
 
 
 
 
 
 
 
 
 
 
 
 
 
 
 
 
 
 
 
 
 
 
 
 
 
 
 
 
 
 
 
 
 
 
 
 
 
 
 
 
 
 
 
 
 
 
 
 
 
 
 
----------------------------------------------------------
 
No menu item available.
Please enter -1 to revert
to the previous menu.
 
 
 
 
 
 
 
 
 
 
 
 
 
 
 
 
 
 
 
 
 
 
 
 
 
 
 
 
 
 
 
 
 
 
 
 
 
 
 
 
 
 
 
 
 
 
 
 
 
----------------------------------------------------------
THE MATERIAL BELOW WAS NOT USED
 
If Assumptions Fail--Transformational Analysis
   BOX-COX HOMOSCEDASTICITY PLOT Y X
   BOX-COX NORMALITY PLOT Y
   BOX-COX LINEARITY PLOT Y X
 
Robust & Distribution-Free Analysis
   Robust Analysis
   Distribution-Free Analysis
 
Statistical graphics
Summary graphics
 
Graphical data analysis/EDA
Graphical residual analysis
Set multiplot
Capture next plot only
Capture all succeeding plots
Set graphics device
Examples
Show graphics gallery
Help
~ANOVA~
~Distributional Analysis~
~EDA~
~Estimation~
~Fitting~
~Gallery~
~Graphical~
~Interlab~
~Location Est./Test.~
~Multivatiate~
~Non-parametric~
~Reliability~
~Random Numbers~
~Residual Analysis~
~Robust Analysis~
~Smoothing~
~SPC~
~Subset Analyses~
~Summaries~
~Testing~
~Times Series~
~Transformations~
~Underlying Assumptions~
~Variation Est./Test.~
~Your First Analysis~
~Your First Fit~
~Your First Histogram~
~Your First Non-Linear Fit~
~Your First Time Series Analysis~
~Your First Weibull Plot~
Your First ...
Fit
Stat Computation
Variable Transformation
5.----------Probability----------
~Gallery~
Your First ...
Probability Calculation
6.----------Mathematics----------
~Arithmetic Operations~
~Built-in Functions~
~Complex Arithmetic~
~Complex Roots~
~Concatonating Functions~
~Convolution~
~Creating Functions~
~Defining Functions~
~Differentiation~
~Diff. Eq. Solver~
~Fitting Functions~
~FFT~
~Fractal Plots~
~Functions~
~Gallery~
~Integration~
~Logical Set Operations~
~Matrix Operations~
~Operations~
~Plotting Functions~
~Poincare Plots~
~Polynomial Operations~
~Roots~
~Set Operations~
~Simplex LP Solver~
~Your First FFT~
Your First ...
Root Calculation
Matrix Operation
Fast Fourier Transform
///////
FRACTAL PLOT
... FREQUENCY PLOT
PLOT
 
 
Summary statistics
Distributional statistics
Graphical data analysis/EDA
Generate data internally
Define variables
Define parameters
Define functions/strings
Evaluate a function
Compute an expression
Create a new variable
Quantitative Statistics
Error Bar Analysis
ERROR BAR PLOT Y X
 
What Process Inspection Plan to Use?
Acceptance Sampling
   Introduction
   OC (Operating Characteristic) Curves
   Sampling Plans
      By Quality Index
         AQL (Acceptance Quality Level) Plans
            MIL-STD-105D
            MIL-STD-414
         LTPD (Lot Tolerance Percent Defective) Plans
         AOQL (Average Outgoing Quality Limit) Plans
      By Inspection Type
         Variable
         Attribute
      By Number of Stages
         Single-Stage
         Double-Stage
         Multi-Stage
   Minimizing Inspection Costs
   Minimizing Repair Costs
 
Business Graphics
   PIE CHART Y
   HISTOGRAM Y
   PARETO PLOT Y
   PLOT
 
Analysis of 2-way Tables
Contingency Tables
Analysis of Covariance
/////
Assumptions           --What are Underlying Assumptions of a Proc.?
Process Summarization --How Summarize the Output from a Process?
Location Analysis     --What is the "Typical Value" of a Process?
Variation Analysis    --What is the Precision of a Process?
Randomness Analysis   --How Random/Un-autocorrelated is the Process?
Stability Analysis    --Is Process Same for All Levels of a Factor?
Uncertainty Analysis  --How Attach Error Bars to an Estimator?
Stat. Proc. Control   --Is the Process Statistically "In Control"?
Reliability Analysis  --What is Process/Product Lifetime?
Warranty Analysis     --How Determine a Warrenty Value for a Product?
Proc. Capability Anal.--How "Good"/Capable is the Process?
Quality Analysis      --How Can the Process be Improved?
Time Series Analysis  --How Analyze Equi-spaced Data?
Regression Analysis   --How Determine the Model in Y = f(X1,X2,...)?
Multifactor Analysis  --What Factors Affect Process Yield?
Multivariate Analysis --How Examine Multiple Responses?
Interlab Analysis     --Are All Labs Equally Good?
500.      0           0. Initial Menu
600.      1               1. Univariate Analysis
1028.     1.1                   1. General Discussion
1037.     1.2                   2. Computing Summary/Descriptive Statistics
1051.     1.2.1                       1. Location Estimation
1078.     1.2.2                       2. Variation (Scale) Estimation
1106.     1.2.3                       3. Skewness Estimation
1133.     1.2.4                       4. Tail Length Estimation
1156.     1.2.5                       5. Autocorrelation estimation
1182.     1.3                   3. Determining General Distributional Characteri
1201.     1.3.1                       1. Frequency Tabulation
490.      1.3.2                       2. Histograms and Cumulative Histograms
490.      1.3.3                       3. Stem and Leaf Diagrams
490.      1.3.4                       4. Frequency Plots and Cumulative Frequenc
490.      1.3.5                       5. Percent Point Plots
490.      1.3.6                       6. Pie Charts
490.      1.4                   4. Selecting a "Good-Fitting" Distribution
490.      1.4.1                       1. Probability Plots
490.      1.4.2                       2. PPCC Plots
490.      1.4.3                       3. Maximum Likelihood Estimation
490.      1.5                   5. Estimating the Parameters of the Distribution
490.      1.5.1                       1. Maximum Likelihood Estimation
490.      1.5.2                       2. Robust Estimates
490.      1.6                   6. Assessing the Goodness of Fit of a Distributi
490.      1.6.1                       1. Superimposing Probability Density Funct
490.      1.6.2                       2. Superimposing Root Density Function and
490.      1.6.3                       3. Probability Plot
490.      1.6.4                       4. Chi-Squared Test
490.      1.6.5                       5. Kolmogorov-Smirnoff Test
490.      1.7                   7. Testing Underlying Assumptions
490.      1.7.1                       1. 4-Plot Analysis
490.      1.8                   8. Testing for Randomness
490.      1.8.1                       1. Lag Plot
490.      1.8.2                       2. Runs Analysis
490.      1.8.3                       3. Distribution-Free Tests
490.      1.8.4                       4. Autocorrelation Plot
490.      1.8.5                       5. Spectral Plot
490.      1.9                   9. Testing for Fixed Location (No Shifts in Loca
490.      1.9.1                       1. t Test
490.      1.9.2                       2. Distribution-Free Tests
490.      1.10                 10. Testing for Fixed Variation (Homoscedasticity
490.      1.10.1                      1. Homoscedasticity Plot
490.      1.11                 11. Transforming to Homoscedasticity
490.      1.11.1                      1. Box-Cox Homoscedasticity Plot
490.      1.11.2                      2. Chi-squared Tests
490.      1.12                 12. Testing for Fixed Distribution
490.      1.12.1                      1. Bihistogram
490.      1.12.2                      2. 4-Plot Analysis
490.      1.12.3                      3. Distribution-Free Tests
490.      1.12.4                      4. Homoscedasticity Plot
490.      1.13                 13. Testing for Symmetry
490.      1.13.1                      1. Symmetry Plot
490.      1.14                 14. Transforming to Symmetry
490.      1.14.1                      1. Box-Cox Symmetry Plot
490.      1.15                 15. Testing for Normality
490.      1.15.1                      1. Normal Probability Plot
490.      1.15.2                      2. Tukey PPCC Plot
490.      1.15.3                      3. t PPCC Plot
490.      1.16                 16. Transforming to Normality
490.      1.16.1                      1. Box-Cox Normality Plot
490.      1.17                 17. Testing for Normal Outliers
490.      1.18                 18. Computing Confidence Limits for Distributiona
490.      1.19                 19. Hypothesis Testing on Distributional Paramete
1500.     2               2. Time Series Analysis (1 Variable)
490.      2.1                   1. General discussion
490.      2.2                   2. Checking for Time-Domain Structure
490.      2.2.1                       1. Run Sequence Plot
490.      2.2.2                       2. Lag Plot
490.      2.2.3                       3. Autocorrelation Plot
490.      2.2.4                       4. Partial-Autocorrelation Plot
490.      2.2.5                       5. Complex Demodulation Plots
490.      2.3                   3. Checking for Frequency-Domain Structure
490.      2.3.1                       1. Spectral Plot
490.      2.3.2                       2. Periodogram
490.      2.4                   4. Checking for Time and Frequency Domain Struct
490.      2.4.1                       1. 4-Plot Analysis
490.      2.5                   5. Testing White Noise (Randomness)
490.      2.5.1                       1. Lag Plot
490.      2.5.2                       2. Runs Analysis
490.      2.5.3                       3. Distribution-Free Tests
490.      2.5.4                       4. Autocorrelation Plot
490.      2.5.5                       5. Spectral Plot
490.      2.5.6                       6. 4-Plot Analysis
490.      2.6                   6. Checking for Trends
490.      2.6.1                       1. Run Sequence Plot
490.      2.6.2                       2. Correlation With Time
490.      2.6.3                       3. Linear Fit Over Time
490.      2.7                   7. Fitting Box-Jenkins Models
490.      2.7.1                       1. Lag Plot
490.      2.7.2                       2. Autocorrelation Plot
490.      2.7.3                       3. Partial Autocorrelation Plot
490.      2.8                   8. Smoothing
490.      2.8.1                       1. Moving Average Smoothing
490.      2.8.2                       2. Least Squares Smoothing
490.      2.8.3                       3. Median Smoothing
490.      2.8.4                       4. Robust Smoothing
490.      2.8.5                       5. Exponential Smoothing
490.      2.8.6                       6. Assessing the Goodness of the Smoothing
490.      2.8.6.1                           1. Residual Standard Deviation
490.      2.8.6.2                           2. Superimposing Raw Data and Fitted
490.      2.8.6.3                           3. Scatter Plots of Residuals
490.      2.8.6.4                           4. Normal Probability Plot of Residu
490.      2.8.6.5                           5. 5-Plot of Residuals
490.      2.9                   9. Filtering
490.      2.9.1                       1. Low-Pass Filters
490.      2.9.2                       2. High-Pass Filters
490.      2.9.3                       3. Assessing the Goodness of the Filtering
1700.     3               3. Time Series Analysis--2 Variables
490.      3.1                   1. General Discussion
490.      3.2                   2. Checking for Time-Domain Structure
490.      3.2.1                       1. Scatter Plot
490.      3.2.2                       2. Multi-Trace Plots
490.      3.2.3                       3. Cross-Spectral Plot
490.      3.2.4                       4. Bihistogram
490.      3.3                   3. Checking for Frequency-Domain Structure
490.      3.3.1                       1. Cross-Spectrum
490.      3.3.2                       2. Coherency Spectrum
490.      3.3.3                       3. Quadrature Spectrum
490.      3.3.4                       4. Co-Spectrum
490.      3.3.5                       5. Gain Spectrum
490.      3.3.6                       6. Argand Spectrum
490.      3.4                   4. Checking for Time and Frequency Domain Struct
490.      3.4.1                       1. 4-Plot Analysis
1800.     4               4. Correlation Analysis
490.      4.1                   1. General Discussion
490.      4.2                   2. Multi-Scatter Plots
490.      4.3                   3. Multi-ANOP Plots
490.      4.4                   4. Multi-Box Plots
490.      4.5                   5. Cross-Correlation Tabulation
490.      4.6                   6. Transforming Variables
490.      4.7                   7. Distribution-free Tests
1900.     5               5. Fitting (1 Independent Variable)
1917.     5.1                   1. General discussion
1954.     5.2                   2. Selecting a Model
1979.     5.2.1                       1. Plotting the Data
2009.     5.2.2                       2. Generate Reference Data Plots
2562.     5.2.2.1                           1. Shape 1--Quadratic
2592.     5.2.2.2                           2. Shape 2--Monotonic up
2622.     5.2.2.3                           3. Shape 3--Monotonic up
2652.     5.2.2.4                           4. Shape 4--S-Shaped
2682.     5.2.2.5                           5. Shape 5--Quadratic
2712.     5.2.2.6                           6. Shape 6--Linear
2742.     5.2.2.7                           7. Shape 7--Square Root
2772.     5.2.2.8                           8. Shape 8--Asymptote
2802.     5.2.2.9                           9. Shape 9--Cubic
2832.     5.2.2.10                         10. Shape 10--Monotonic
2862.     5.2.2.11                         11. Shape 11--Logarithm
2892.     5.2.2.12                         12. Shape 12--Asymptote
2922.     5.2.2.13                         13. Shape 13--Skewed
2952.     5.2.2.14                         14. Shape 14--Skewed
2982.     5.2.2.15                         15. Shape 15--Quadratic
3012.     5.2.2.16                         16. Shape 16--Skewed
3042.     5.2.2.17                         17. Shape 17--Bell-Shaped
3072.     5.2.2.18                         18. Shape 18--Hyberbolic
3102.     5.2.2.19                         19. Shape 19--Neg. Exponential
3132.     5.2.2.20                         20. Shape 20--Z-Shaped
3162.     5.3                   3. Fitting a Model
3174.     5.3.1                       1. Fitting Linear Models
490.      5.3.2                       2. Fitting Polynomial Models
490.      5.3.3                       3. Fitting Non-Linear Models
490.      5.3.4                       4. Fitting Rational Functions
3218.     5.3.5                       5. Fitting Splines
3254.     5.4                   4. Assessing the Goodness of Fit of the Model
490.      5.4.1                       1. Residual Standard Deviation
490.      5.4.2                       2. Lack of Fit F Tests
3297.     5.4.3                       3. Superimposing Raw Data and Fitted Curve
3330.     5.4.4                       4. 4-Plot of Residuals
3366.     5.4.5                       5. Scatter Plots of Residuals
3399.     5.4.5.1                           1. Reference Scatter Plots of Residu
490.      5.4.5.1.1                               1  Residual Scatter Plot--Idea
490.      5.4.5.1.2                               2  Residual Scatter Plot--Wedg
490.      5.4.5.1.3                               3  Residual Scatter Plot--Wedg
490.      5.4.5.1.4                               4  Residual Scatter Plot--Wedg
490.      5.4.5.1.5                               5  Residual Scatter Plot--Line
490.      5.4.5.1.6                               6  Residual Scatter Plot--Quad
490.      5.4.5.1.7                               7  Residual Scatter Plot--Quad
3670.     5.4.5.1.8                               8  Residual Scatter Plot--Spli
490.      5.4.6                       6. Lag Plot of Residuals
490.      5.4.7                       7. Histograms of Residuals
490.      5.4.8                       8. Normal Probability Plot of Residuals
490.      5.5                   5. Improving the Model
490.      5.5.1                       1. Transforming to Simplify the Model
490.      5.5.2                       2. Transforming to Achieve Homogeneity
490.      5.5.3                       3. Transforming to Achieve Normality
490.      5.5.4                       4. Adding New Variables
490.      5.5.5                       5. Changing the Form of the Model
490.      5.6                   6. Fitting With Weights
490.      5.7                   7. Fitting With Constraints
490.      5.7                   8. Fitting With Other Criteria (e.g., L1 Fitting
4000.     6               6. Fitting (2 or More Independent Variable)
4018.     6.1                   1. General discussion
490.      6.2                   2. Selecting Variables for To Be Included in the
490.      6.2.1                       1. Multi-Run Sequence Plots
490.      6.2.2                       2. Multi-Histograms
490.      6.2.3                       3. Multi-Scatter Plots
490.      6.2.4                       4. Multi-ANOP Plots
490.      6.2.5                       5. Cross-Correlation Tabulation
490.      6.2.6                       6. Box Plots
490.      6.2.7                       7. Cp Plot
490.      6.3                   3. Selecting a Model
490.      6.3.1                       1. Plotting the Data
490.      6.3.2                       2. Making Use of Reference Curves
490.      6.4                   4. Fitting a Model
490.      6.4.1                       1. Fitting Multi-Linear Models
490.      6.4.2                       2. Fitting Non-Linear Models
490.      6.5                   5. Assessing the Goodness of Fit of the Model
490.      6.5.1                       1. Residual Standard Deviation
490.      6.5.2                       2. Lack of Fit F Tests
490.      6.5.3                       3. Superimposing Raw Data and Fitted Curve
490.      6.5.4                       4. Scatter Plots of Residuals
490.      6.5.5                       5. Normal Probability Plot of Residuals
490.      6.5.6                       6. 4-Plot of Residuals
490.      6.6                   6. Improving the Model
490.      6.6.1                       1. Transforming to Simplify the Model
490.      6.6.2                       2. Transforming to Achieve Homogeneity
490.      6.6.3                       3. Transforming to Achieve Normality
490.      6.6.4                       4. Adding New Variables
490.      6.6.5                       5. Changing the Form of the Model
490.      6.7                   7. Fitting With Weights
490.      6.8                   8. Fitting With Constraints
490.      6.8                   8. Fitting With Other Criteria (e.g., L1 Fitting
4200.     7               7. ANOVA Modeling
490.      7.1                   1. General discussion
490.      7.2                   2. Selecting Variables for To Be Included in the
490.      7.2.1                       1. Multi-Run Sequence Plots
490.      7.2.2                       2. Multi-Histograms
490.      7.2.3                       3. Multi-Scatter Plots
490.      7.2.4                       4. Multi-ANOP Plots
490.      7.3                   3. Examining 1-Factor Models
490.      7.3.1                       1. 1-Way ANOVA
490.      7.3.2                       2. 1-Way GANOVA
490.      7.3.3                       3. Scatter Plots
490.      7.3.4                       4. Box Plot
490.      7.3.5                       5. ANOP Line Plot
490.      7.3.6                       6. ANOP Character Plot
490.      7.3.7                       7. I Plot
490.      7.3.8                       8. Distribution-free Tests
490.      7.3.9                       9. Correlation
490.      7.3.10                     10. Categorical Data Analauysis
490.      7.3.11                     11. Cross-Corrleation
490.      7.3.12                     12. Discrete Contour Plot
490.      7.3.13                     13. Frequency Tabulation
490.      7.3.14                     14. Cross-Tabulation
490.      7.3.15                     15. Chi-squared
490.      7.4                   4. Examining 2-Factor Models
490.      7.4.1                       1. 2-Way ANOVA
490.      7.4.2                       2. 2-Way GANOVA
490.      7.4.3                       3. Median Polish
490.      7.4.4                       4. Multi-Trace Plots
490.      7.4.5                       5. 3-D Plot
490.      7.4.6                       6. Spike Plots
490.      7.5                   5. Examining 3-Factor Models
490.      7.5.1                       1. 3-Way ANOVA
490.      7.5.2                       2. 3-Way GANOVA
490.      7.5.3                       3. Multi-Cell Plots
490.      7.6                   6. Examining 4-factor Models
490.      7.6.1                       1. 4-Way ANOVA
490.      7.6.2                       2. Multi-Plot 2-Way GANOVA
490.      7.6.3                       3. Multi-Plot 3-Way GANOVA
490.      7.7                   7. Examining 5-Factor Models
490.      7.7.1                       1. 5-Way ANOVA
490.      7.7.2                       2. Multi-Plot 3-Way GANOVA
490.      7.8                   8. Examining 1-Factor Models With Only 2 Treatme
490.      7.8.1                       1. 1-Way ANOVA
490.      7.8.2                       2. t Test
490.      7.8.3                       3. Bihistogram
490.      7.9                   9. Examining 2**k Models
490.      7.9.1                       1. Square Plots, Cube Plots, etc.
490.      7.10                 10. Assessing the Goodness of Fit of the Model
490.      7.10.1                      1. Residual Standard Deviation
490.      7.10.2                      2. Lack of Fit F Tests
490.      7.10.3                      3. GANOVA, Parallelism, and Non-Additivity
490.      7.10.4                      4. Superimposing Raw Data and Fitted Curve
490.      7.10.5                      5. Scatter Plots of Residuals
490.      7.10.6                      6. Normal Probability Plot of Residuals
490.      7.10.7                      7. 4-Plot of Residuals
490.      7.11                 11. Improving the Model
490.      7.11.1                      1. Residual Standard Deviations For Sub-Mo
490.      7.11.2                      2. F Tests For Sub-Models
490.      7.11.3                      3. Transforming to Simplify the Model
490.      7.11.4                      4. Transforming to Achieve Additivity
490.      7.11.5                      5. Transforming to Achieve Homogeneity
490.      7.11.6                      6. Transforming to Achieve Normality
490.      7.11.7                      7. Omitting Variables From the Model
490.      7.11.7.1                          1. F Tests for Sub-Models
490.      7.11.8                      8. Selecting Additional Variables For the
490.      7.11.8.1                          1. Scatter Plots of Residuals on New
490.      7.11.8.2                          2. Box Plots of Residuals on New Var
490.      7.11.9                      9. Changing the Form of the Model
4600.     8               8. Multivariate Analysis
490.      8.1                   1. General Discussion
490.      8.2                   2. Cluster Analysis
490.      8.3                   3. Discriminant Analysis
490.      8.4                   4. Principal Component Analysis
490.      8.5                   5. Canonical Analysis
490.      8.6                   6. Testing Multivariate Normality--Q-Q Plot
490.      8.5
4700.     9               9. Probability Analysis
490.      9.1                   1. General Discussion
490.      9.2                   2. Generating Random Numbers/Simulation/Monte Ca
490.      9.3                   3. Computing Percent Points
490.      9.4                   4. Computing Probability Density Functions
490.      9.5                   5. Computing Cumulative Distribution Functions
490.      9.6                   6. Plotting Percent Points
490.      9.7                   7. Plotting Probability Density Functions
490.      9.8                   8. Plotting Cumulative Distribution Functions
490.      9.9                   9. Superimposing Probability Density Funtions on
4800.     10             10. Quality Control
490.      10.1                  1. General Discussion
490.      10.2                  2. Testing for Trends
490.      10.2.1                      1. Run Sequence Plot
490.      10.2.2                      2. Mean Control Chart
490.      10.3                  3. Testing for Shifts in Location
490.      10.3.1                      1. Run Sequence Plot
490.      10.3.2                      2. Mean Control Chart
490.      10.4                  4. Testing for Shifts in Variation
490.      10.4.1                      1. Range Control Chart
490.      10.4.2                      2. Standard Deviation Control Chart
490.      10.5                  5. Testing for Outliers
490.      10.6                  6. Interlaboratory Testing
490.      10.6.1                      1. Youden Plots
4900.     11             11. Distribution-Free Analysis
490.      11.1                  1. General Discussion
490.      11.2                  2. Testing for Randomness
490.      11.2.1                      1. Runs Analysis
490.      11.2.2                      2. Sign Test
490.      11.2.3                      3. Median Test
490.      11.3                  2. Testing for Fixed Location (No Shifts)
490.      11.3.1                      1. Sign Test
490.      11.4                  3. Testing for Fixed Variation (Homoscedasticity
490.      11.4.1                      1. Sign Test on First Differences
490.      11.5                  4. Testing for Goodness of Fit of a Distribution
490.      11.5.1                      1. Kolmogorov-Smirnoff Test
490.      11.6                  5. Testing for Correlation
490.      11.6.1                      1. Rank Correlation Coefficient
490.      11.7                  6. ANOVA Modeling
490.      11.7.1                      1. ANOVA On Ranks
490.      11.7.2                      2. Mann-Whitney Tests
5000.     12             12. Robust Analysis
490.      12.1                  1. General Discussion
490.      12.2                  2. Univariate Estimation
490.      12.2.1                      1. Robust Location Estimation
490.      12.2.2                      2. Robust Scale Estimation
490.      12.3                  2. Smoothing
490.      12.3.1                      1. Median Smoothing
490.      12.3.2                      2. Robust Smoothing
490.      12.4                  4. Fitting
490.      12.4.1                      1. L1 Fitting
490.      12.5                  5. ANOVA Modeling
490.      12.5.1                      1. ANOVA On Ranks
490.      12.5.2                      2. Median Polish
5100.     13             13. Exploratory Data Analysis
490.      13.1                  1. General Discussion
490.      13.2                  2. Run Sequence Plots
490.      13.3                  3. Lag Plots
490.      13.4                  4. Stem and Leaf Diagrams
490.      13.5                  5. Histograms
490.      13.6                  6. Normal Probability Plots
490.      13.7                  7. Scatter Plots
490.      13.8                  8. Box Plots
 
 
 
 
 
 
 
 
 
 
 
 
 
 
 
 
 
 
 
 
 
 
 
 
 
 
 
 
 
 
 
 
 
 
 
 
 
 
 
 
 
 
 
 
 
 
 
 
 
 
 
 
 
 
 
 
 
 
 
 
 
 
 
 
 
 
 
 
 
 
 
 
 
 
 
 
 
 
 
 
 
 
 
 
 
 
 
 
 
 
 
 
 
 
 
 
 
 
 
 
 
 
 
 
 
 
 
 
 
 
 
 
 
 
 
 
 
 
 
 
 
 
 
 
 
 
 
 
 
----------------------------------------------------------
 
No menu item available.
Please enter -1 to revert
to the previous menu.
 
 
 
 
 
 
 
 
 
 
 
 
 
 
 
 
 
 
 
 
 
 
 
 
 
 
 
 
 
 
 
 
 
 
 
 
 
 
 
 
 
 
 
 
 
 
 
 
 
